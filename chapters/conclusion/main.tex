\chapter{Conclusion and Further Work}

We consider there to be two main conclusions to be drawn from the experiments presented in \cref{sec:experiments}:
\begin{itemize}
  \item LiDAR input data is essential in order to predict accurate surface normal vector rasters, but RGB can be used in \emph{addition} in order to marginally increase predictive performance.
  \item A multitask model which predicts \emph{both} semantic segmentation masks \emph{and} surface normal vectors ends up being more accurate at segmentation than a respective single-task segmentation model.
\end{itemize}

We have now presented a complete end-to-end pipeline for predicting three-dimensional flat roof surface polygons.
The pipeline can be considered to consist of four sequential components:
%
\begin{enumerate}
  \item \textbf{Feature engineering} -- Construct semantic segmentation masks and surface normal rasters from ground truth roof surface GIS data.
  \item \textbf{Prediction} -- Construct and train CNN architecture(s) for the prediction of semantic segmentation masks and surface normal rasters.
  \item \textbf{Instance clustering} -- Use predicted surface normal rasters in order to partition the semantic segmentation masks.
  \item \textbf{Vectorization} -- Reconstruct three-dimensional polygons from original LiDAR data in combination with the processed instance segmentation masks.
\end{enumerate}
%
The first component, \enquote{feature engineering}, can be considered as the problem formulation, while the remaining three components are constructed in order to solve the problem.
The three \enquote{solution components} are highly modular and composable, and different implementations can easily replace any of them.
We consider this thesis a proof-of-concept of sorts, establishing a rough sketch of the types of building blocks required in order to solve the problem at hand.
Here are some examples of how these building blocks can be improved upon in further work:
%
\begin{itemize}
  \item Implement methods for performing image augmentation during training when surface normal vector rasters are involved.
    Any rotation of the input raster data must be performed on each individual ground truth normal vector.
  \item Utilize newer neural network architectures for the prediction and refinement of semantic segmentation masks and surface normal rasters, e.g.\ GANs.
  \item Investigate the application of different activation functions in the surface normal vector raster output layer.
    Preferably utilize the fact that $\norm{\vec{n}} \equiv 1$ and $n_z \geq 0$ to a greater degree.
  \item Implement other polygon simplification methods which uses more domain-specific knowledge.
    Simplified polygons which share a common border \emph{before} simplification should preferably still share the same border \emph{after} simplification, for instance.
  \item Implement metrics for evaluating the accuracy of the \enquote{instance clustering} and \enquote{vectorization} components.
    Use these metrics in order to perform a proper hyperparameter search for any parameters used by these clustering and vectorization algorithms.
    These metrics can also be used in order to compare different algorithms all together.
\end{itemize}
