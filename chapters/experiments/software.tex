The majority of the source code written in order to produce and present the results in this paper is written in Python as it arguably has the best software ecosystem for both GIS \emph{and} deep learning workflows.
This work would not have been possible if not for the vast array of high quality open source software available.
The Geospatial Data Abstraction Library (GDAL) \cite{dep:gdal} has been extensively used in order to process GIS data, and the python wrappers for GDAL, Rasterio \cite{dep:rasterio} for raster data and Fiona \cite{dep:fiona} for vector data, are central building blocks of the data processing pipeline.
% Further, Numpy \cite{dep:numpy}, Shapely \cite{dep:shapely}, scikit-learn \cite{dep:sklearn} / scikit-image \cite{dep:sklearn}, and GeoPandas \cite{dep:geopandas} have been used in order to shape the data into a final format suitable for machine learning purposes.
The machine learning framework of choice has been the new 2.1 release of TensorFlow \cite{dep:tensorflow}, most of the modelling code having been written with the declarative Keras API.
