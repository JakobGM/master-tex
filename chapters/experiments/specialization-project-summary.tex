The research questions posed in this thesis is a strict superset of the problems solved in my previous specialization project~\cite{specialization-project}, a project which solely focuses on the semantic segmentation of roof structures from remote sensing data.
We will therefore provide a brief summary of the conclusions made in~\cite{specialization-project} before continuing onto the experiments which are new to this thesis.
We refer to the original paper\footnote{Original specialization project PDF files can be found at: \url{https://jakobgm.com/pdf/ntnu_reports/9-project-thesis.pdf}} for a more detailed presentation and analysis of these results.
%
\begin{itemize}
  \item LiDAR raster data is far more suitable for the prediction of semantic roof segmentation masks than aerial RGB photography.
  \item A model which consumes both LiDAR \emph{and} RGB photography has a small, but not negligible, performance advantage over a model which only uses LiDAR.
  \item Regularization techniques such as batch normalization and dropout have been shown to significantly improve the resulting test performance of the trained networks, while data augmentation had an insignificant effect.
  \item When it comes to the normalization of LiDAR input rasters, \cref{alg:metric-normalization} -- \enquote{\texttt{Nodata-aware metric normalization}} was shown to perform slightly better than \cref{alg:local-min-max-scaling} -- \enquote{\texttt{Nodata-aware local min-max normalization}}.
  \item While semantic segmentation models trained with binary cross-entropy loss (BCEL) showed a purely quantitative advantage on almost all test metrics compared to alternative losses, the soft variant losses still showed a greater degree of \enquote{common sense}.
  The soft loss models also showed a greater propensity for ignoring wrong ground truth data during training.
\end{itemize}
%
Based on these results, we will train all models with both dropout and batch normalization.
LiDAR input raster data will be normalized according to \cref{alg:metric-normalization} -- \enquote{\texttt{Nodata-aware metric normalization}} with parametrization $\gamma = 30$.
The semantic segmentation loss function that will be used is the soft Jaccard loss as presented in \cref{eq:soft-jaccard-loss}.
