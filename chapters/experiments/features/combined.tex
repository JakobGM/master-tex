We now construct a model which uses both LiDAR and RGB data in combination in order to produce predictions.
The training procedure of the combined data model is shown in \cref{fig:combined-training}, and the training procedures of the two single-input models have been included for comparison.
%
\begin{figure}[H]
  \includegraphics[width=0.75\textwidth]{img/training/normals_with_lidar_and_rgb+normals_with_only_lidar+normals_with_only_rgb-train+validation-loss.pdf}
  \appcaption{%
    Training procedure for surface normal vector models using LiDAR and/or RGB as input.
  }{%
    The training epochs are given along the horizontal axis, while the end-of-epoch cosine similarity loss evaluations are given along the vertical axis.
    The LiDAR-only model is shown in \textcolor{blue}{blue}, the RGB-only model in \textcolor{orange}{orange}, and finally the combined input model in \textcolor{green}{green}.
    The training split loss is shown with dashed lines, while the validation split is shown with solid lines.
    The best validation loss epochs are annotated with solid circles, with specific validation loss values provided in the figure legend.
  }%
  \label{fig:combined-training}
\end{figure}
\noindent
\Cref{fig:combined-training} does not show any immediately obvious in predictive performance between the LiDAR-only model and the combined input model.
The distribution of the instance-averaged cosine similarities over the test set have been plotted in \cref{fig:combined-test-distribution}.
%
\begin{figure}[H]
  \includegraphics[width=0.75\textwidth]{img/evaluation/normals_with_lidar_and_rgb/instance-averaged-cosine-similarities.pdf}
  \appcaption{%
    Distribution of instance-averaged cosine similarities of the combined input surface normal model over the test split.
  }{%
    See \cref{fig:lidar-test-distribution} for detailed figure description.
  }%
  \label{fig:combined-test-distribution}
\end{figure}
%
Compared to the LiDAR-only model, the combined input model has decreased the mean instance-averaged cosine similarity from \num{0.00294} to \num{0.00265}, and negligible increased the median from \num{0.00043} to \num{0.00044}.
With other words, the combined input model has not increased the predictive performance as much as the LiDAR-only model improved upon the RGB-only model, but it still has somewhat better test performance.

The median test prediction of the combined input model is provided in \cref{fig:combined-median-prediction}.
The result of post-processing the median test prediction is presented in \cref{fig:combined-median-processing}.
%
\begin{figure}
  \normalresult{normals_with_lidar_and_rgb}{38644}{0}
  \appcaption{%
    Median test prediction for the combined input normal vector model.
  }{%
    See \cref{fig:lidar-median-prediction} for detailed figure description.
    % Mask fraction \num{0.0882721}, 21 instances, loss = \num{0.911767}, average cosine similarity = \num{0.000442978}, standard deviation = \num{0.00205065}.
  }%
  \label{fig:combined-median-prediction}
\end{figure}
\begin{figure}
  \polygonresult{normals_with_lidar_and_rgb}{ground_truth}{38644}{0}
  \appcaption{%
    Post-processing of median test prediction for combined-input model with respect to the cosine similarity loss component.
  }{%
    See \cref{fig:lidar-median-processing} for a detailed figure description.
  }%
  \label{fig:combined-median-processing}
\end{figure}
