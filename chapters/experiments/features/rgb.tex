A model using only RGB data is trained, and the training procedure is summarized in \cref{fig:rgb-training} side-by-side the LiDAR-only model for comparison purposes.
%
\begin{figure}[H]
  \includegraphics[width=0.75\textwidth]{img/training/normals_with_only_lidar+normals_with_only_rgb-train+validation-loss.pdf}
  \appcaption{%
    Training procedure for surface normal vector models using \emph{either} LiDAR or RGB as input.
  }{%
    The training epochs are given along the horizontal axis, while the end-of-epoch cosine similarity loss evaluations are given along the vertical axis.
    The LiDAR-only model is shown in \textcolor{blue}{blue}, while the RGB-only model is shown in \textcolor{orange}{orange}.
    The training split loss is shown with dashed lines, while the validation split is shown with solid lines.
    The best validation loss epochs are annotated with solid circles, with specific validation loss values provided in the figure legend.
  }%
  \label{fig:rgb-training}
\end{figure}
\noindent
From \cref{fig:rgb-training} it is immediately obvious that a RGB-only model performs worse than a LiDAR-only model, as hypothesized.
In order to confirm this, the instance-averaged cosine similarity distribution over the test set is yet again plotted in \cref{fig:rgb-test-distribution}, this time for the RGB-only model instead.
%
\begin{figure}
  \includegraphics[width=0.75\textwidth]{img/evaluation/normals_with_only_rgb/instance-averaged-cosine-similarities.pdf}
  \appcaption{%
    Distribution of instance-averaged cosine similarities of the RGB-only surface normal model over the test split.
  }{%
    See \cref{fig:lidar-test-distribution} for detailed figure description.
  }%
  \label{fig:rgb-test-distribution}
\end{figure}
%
As can be seen in \cref{fig:rgb-test-distribution} the mean instance-averaged cosine similarity evaluated over the test set increases from \num{0.00294} to \num{0.00402}, and the median increases from \num{0.00043} to \num{0.00078}.
The RGB-only surface normal vector model can therefore be concluded to be strictly worse than the LiDAR-only equivalent model.

Yet again we plot the median test prediction of the RGB-only model in \cref{fig:rgb-median-prediction} as a representative prediction.
The result of post-processing the median test prediction is presented in \cref{fig:rgb-median-processing}.
%
\begin{figure}
  \normalresult{normals_with_only_rgb}{4641}{0}
  \appcaption{%
    Median test prediction for RGB-only normal vector model.
  }{%
    The LiDAR DSM input, $\lidarraster$, has not been used by the model, but is provided for better visual insight.
    See \cref{fig:lidar-median-prediction} for detailed figure description.
    % Mask fraction \num{0.114014}, 18 instances, loss = \num{0.886085}, average cosine similarity = \num{0.000867788}, standard deviation = \num{0.00410062}.
  }%
  \label{fig:rgb-median-prediction}
\end{figure}
\begin{figure}
  \polygonresult{normals_with_only_rgb}{ground_truth}{4641}{0}
  \appcaption{%
    Post-processing of median test prediction for RGB-only model with respect to the cosine similarity loss component.
  }{%
    See \cref{fig:lidar-median-processing} for a detailed figure description.
  }%
  \label{fig:rgb-median-processing}
\end{figure}
