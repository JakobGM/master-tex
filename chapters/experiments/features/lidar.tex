We start by training a model based solely on LiDAR data normalized according to \cref{alg:metric-normalization}.
The training procedure is presented in \cref{fig:lidar-training}.
%
\begin{figure}[H]
  \includegraphics[width=0.75\textwidth]{img/training/normals_with_only_lidar-train+validation-loss.pdf}
  \appcaption{%
    Training procedure of LiDAR-only U-Net-derived architecture for predicting surface normals over 100 training epochs.
  }{%
    The training epochs are given along the horizontal axis, while the end-of-epoch cosine similarity loss evaluations are given along the vertical axis.
    Validation split loss is shown as a \textcolor{blue}{blue} solid line, while the training split loss is shown as a \textcolor{blue}{blue} dashed line.
    The epoch yielding the best validation loss is annotated as a solid \textcolor{blue}{blue} circle, in this case the 99th epoch with a validation loss of \num{13.1353}.
  }%
  \label{fig:lidar-training}
\end{figure}
\noindent
Notice how \cref{fig:lidar-training} shows that the model has a validation loss which is consistently lower than the training loss.
The \textit{tile-averaged}, \textit{batch-summed} cosine similarity loss used during training is highly data-dependent, portraying great variance across tiles, and consequently the data splits, as shown in this figure.
We can therefore not necessarily conclude that the model performs better on the validation split than the training split.

\Cref{fig:lidar-test-distribution} shows the distribution of the \textit{instance-averaged cosine similarities} over the test split.
The instance-averaged cosine similarity for a given instance is defined as the average cosine similarity between the predicted normal vectors (over the respective ground truth \emph{instance} mask) and the ground truth normal vector of the given surface instance.

\begin{figure}[H]
  \includegraphics[width=0.75\textwidth]{img/evaluation/normals_with_only_lidar/instance-averaged-cosine-similarities.pdf}
  \appcaption{%
    Distribution of instance-averaged cosine similarities of the LiDAR-only surface normal model over the test split.
  }{%
    The instance-average cosine similarity is given along the horizontal axis, while the frequency of each instance-averaged cosine similarity is given along the log-scaled vertical axis.
    The mean of the instance-averaged cosine similarities is shown as an solid \textcolor{orange}{orange} line, while the median is annotated as a solid \textcolor{green}{green} line.
    The interquartile range is annotated as dashed \textcolor{green}{green} lines.
    Specific statistical values are provided in the figure legend.
  }%
  \label{fig:lidar-test-distribution}
\end{figure}

The median-performing prediction with respect to the average cosine similarity is presented in \cref{fig:lidar-median-prediction} as a representative model prediction.
That is, half of the model predictions over the test set perform \emph{worse} than the prediction presented in \cref{fig:lidar-median-prediction}, while the other half perform \emph{better}.

\begin{figure}[H]
  \normalresult{normals_with_only_lidar}{12858}{2}
  \appcaption{%
    Median test prediction for LiDAR-only normal vector model.
  }{%
    Each image tile represents the following: \vspace{0.5em}\\
    $\lidarraster$ (upper left) -- LiDAR DSM raster. \\
    $\rgbraster$ (lower left) -- Aerial RGB photography. \\
    $\prednormraster$ (upper middle) -- Predicted surface normal vector raster. \\
    $\segraster \odot \prednormraster$ (lower middle) -- Segmented predicted surface normal vector raster using the ground truth semantic segmentation mask. \\
    $\normraster$ (upper right) -- Ground truth surface normal vector raster. \\
    $\mathcal{L}$ (lower right) -- Pixel-wise cosine similarity. \vspace{0.5em}\\
    The aerial RGB photography input, $I$, has not been used by the model, but is provided for better visual insight.
    % Mask fraction \num{0.113434}, 30 instances, loss = \num{0.11378}, average cosine similarity = \num{0.000452751}, standard deviation = \num{0.00208622}.
  }%
  \label{fig:lidar-median-prediction}
\end{figure}

The result of post-processing the median test prediction is presented in \cref{fig:lidar-median-processing}.
\begin{figure}[H]
  \polygonresult{normals_with_only_lidar}{ground_truth}{12858}{2}
  \appcaption{%
    Post-processing of median test prediction for LiDAR-only model with respect to the cosine similarity loss component.
  }{%
    Each image tile represents the following: \vspace{0.5em}\\
    $\predsegnormraster$ (upper left) -- Predicted segmented normal vector raster, using the ground truth segmentation mask, $\segraster \odot \prednormraster$. \\
    $\normraster$ (upper right) -- Ground truth normal vector raster. \\
    $\texttt{DBSCAN}\left(\predsegnormraster\right)$ (middle left) -- DBSCAN-clustered normal vectors, black indicating noisy outliers. \\
    $\texttt{KNN}\left(\texttt{DBSCAN}\left(\predsegnormraster\right)\right)$ (middle right) -- Result of applying KNN-clustering to the DBSCAN-identified noise. \\
    $\predpolygons$ (lower left) -- Unsimplified predicted polygons. \\
    $\simplifiedpolygons$ (lower right) -- Simplified predicted polygons. \vspace{0.5em}\\
  }%
  \label{fig:lidar-median-processing}
\end{figure}
