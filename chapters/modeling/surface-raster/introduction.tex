The previous section described the deep learning approach for predicting \emph{semantic} segmentation masks.
In our specific case, these semantic segmentation masks denote which pixels contain roof structures, and which do not.
Such semantic segmentation masks are not sufficient for achieving our stated goal of reconstructing three-dimensional roof surface geometries from remote sensing data.
In order to achieve this goal we must predict \emph{instance} segmentation masks instead, that is, not only determining if a given pixel contains a roof structure, but also determining which specific flat roof surface the pixel is part of.
This instance segmentation mask can be subsequently post-processed (as outlined in \cref{chap:post-processing}) in order to produce three-dimensional vector polygons representing flat roof surfaces.

In the upcoming \cref{chap:post-processing} - \enquote{\nameref{chap:post-processing}}, we will present a procedure for partitioning semantic segmentation maps into instance segmentation maps by clustering normal vectors.
This section will describe the prediction of these normal vectors.
Since this is a relatively novel approach to predicting instance segmentation masks, there exists relatively little existing work in the academic literature which is directly applicable.
We will, however, present some existing work which is tangentially relevant to our task at hand in \cref{sec:normal-vector-literature}.
A U-Net derived CNN architecture for predicting rasterized surface normal vectors is presented in \cref{sec:normal-model}.
