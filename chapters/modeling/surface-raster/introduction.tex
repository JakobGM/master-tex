The previous section described the deep learning approach for predicting \emph{semantic} segmentation masks.
In our specific case, these semantic segmentation masks denote which pixels contain roof structures, and which do not.
Such semantic segmentation masks are not sufficient for achieving our stated goal of reconstructing three-dimensional roof surface geometries from remote sensing data.
In order to achieve this goal we must produce \emph{instance} segmentation masks instead, that is, not only determining if a given pixel contains a roof structure, but also determining which specific flat roof surface the pixel is part of.
% We will \emph{indirectly} produce instance segmentation masks by predicting surface normal vector rasters which can be subsequently post-processed in combination with semantic segmentation masks (as outlined in \cref{chap:post-processing}) in order to produce instance segmentation masks.

In the upcoming \cref{chap:post-processing} - \enquote{\nameref{chap:post-processing}}, we will present a procedure for partitioning semantic segmentation maps into instance segmentation maps by clustering normal vectors.
Subsequent post-processing steps can be applied in order to convert these instance segmentation maps into three-dimensional vector polygons representing flat roof surfaces.
This section will describe the prediction of these surface normal vector rasters which are used by these post-processing steps.
Since this is a relatively novel approach to producing instance segmentation masks, there exists relatively little existing work in the academic literature which is directly applicable.
We will, however, present some existing work which is tangentially relevant to our task at hand in \cref{sec:normal-vector-literature}.
A U-Net derived CNN architecture for predicting rasterized surface normal vectors is presented in \cref{sec:normal-model}.
