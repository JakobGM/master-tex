In order to describe segmentation metrics, it is useful to define the following quantities:

\begin{center}
  \fbox{%
    \parbox{0.9\linewidth}{%
      \begin{description}[leftmargin=1cm, before={\renewcommand\makelabel[1]{\bfseries ##1:}}]
        \item[Condition Positive (P)]
          Number of object class pixels in ground truth mask.
        \item[Condition Negative (N)]
          Number of non-object class pixels in ground truth mask.
        \item[True Positive (TP)]
          Number of pixels correctly predicted as being part of object class (correctly identified).
        \item[True Negative (TN)]
          Number of pixels correctly predicted as \textit{not} being part of object class (correctly rejected).
        \item[False Positive (FP)]
          Number of pixel incorrectly predicted as being part of object class (incorrectly identified).
        \item[False Negative (FN)]
          Number of pixel incorrectly predicted as \textit{not} being part of object class (incorrectly rejected).
      \end{description}
    }
  }
\end{center}
%
False positives (FP) are often knows as \textit{type I errors} in statistics, and false negatives (FN) as \textit{Type II errors}.
The greater the values of TP and TN, the better, and the smaller the values of FP and FN, the better.
A visual representation of these classifications is given in \cref{fig:confusions}.

\begin{figure}[H]
  \includegraphics[width=\linewidth]{confusions}
  \caption[Illustration of TP, TN, FP, and FN for a binary segmentation prediction.]{%
    Binary segmentation problem of size $256 \times 256$.
    The ground truth, a rectangle of size $120 \times 80$ is shown on the left.
    The \enquote{predicted} mask, shown in the middle, is of the same size, but offset by $(-30, -30)$.
    The right figure shows the visual equivalent of a confusion matrix.
    True positives are shown in \textcolor{tp}{dark blue}, true negatives in \textcolor{gray}{light gray}, false positives in \textcolor{fp}{green}, and false negatives in \textcolor{fn}{red}.
  }%
  \label{fig:confusions}
\end{figure}

The simplest metric for semantic segmentation is the \textit{pixel accuracy} metric.
This metric simply reports the percentage of pixels that were correctly classified.
More formally, it can be defined as:
%
\begin{equation*}
    \textbf{accuracy}
    \defeq
    \frac{TP + TN}{TP + TN + FP + FN} = \frac{TP + TN}{P + N}
\end{equation*}
%
The problem with the pixel-wise accuracy metric is that it does not take class imbalances into account.
Consider a problem where 95\% of all pixels are considered to be of class $0$, and the remaining 5\% of class $1$.
If we construct a model which predicts $0$ regardless of the feature inputs provided to the model, the model will achieve a 95\% accuracy score.
This makes pixel-wise accuracy scores hard to interpret when you do not know the class balance of the respective dataset and the accuracy grouped by class.
This is why it is often replaced by other metrics which take imbalances into account.
A pair of such metrics are \textit{sensitivity} and \textit{specificity}, formally defined as:
%
\begin{small}
\begin{align*}
    \textbf{sensitivity}
    &=
    \frac{\text{number of true positives}}{\text{number of true positives + number of false negatives}}
    =
    \frac{TP}{TP + FN}
    =
    \frac{TP}{P}
    \\
    \textbf{specificity}
    &=
    \frac{\text{number of true negatives}}{\text{number of true negatives + number of false positives}}
    =
    \frac{TN}{TN + FP}
    =
    \frac{TN}{N}
\end{align*}
\end{small}
%
The \textit{sensitivity} is therefore a measure of how good a given model prediction is able to identify positives as a relative, fractional value.
Likewise, the \textit{specificity} is a measure of how good a given model prediction is able to identify negatives.
