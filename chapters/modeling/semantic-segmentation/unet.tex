We have chosen the U-Net architecture for segmenting roof structures and we will present numerical experiments in \cref{sec:experiments}.
The U-Net model has already been briefly described in the previous section and the architecture has been illustrated in \cref{fig:unet}, but we will provide a more detailed summary of the U-Net architecture here.
An alternative visual representation of the U-Net architecture is provided in \cref{fig:unet2}.
The U-Net architecture consists of four sequential \enquote{encoder modules}, each module applying a set number of convolutional filters followed by the application of the ReLU activation function.
The number of trained convolutional filters in each encoder module is respectively: 64, 128, 256, and 512.
Each module ends with a downsampling operation in form of max-pooling of size 2.
Since our input images have resolution $256 \times 256$, we end up with inputs of size $16 \times 16$ to the \enquote{bottleneck convolution module} where 1024 convolutional filters are trained.
The bottleneck convolution module is placed at the bottom of the U-shape in \cref{fig:unet2}.
Each decoder block utilizes batch normalization and max-pooling dropout.
The \enquote{decoder modules} apply transposed convolutions in order to upsample the resolution by a factor of two, the number of filters being equivalent to their respective \enquote{mirror encoders}, i.e.\ the encoder modules handling inputs with identical resolutions.
Four such modules are applied in order to yield a final output resolution of size $256 \times 256$, the original input resolution.
The outputs of the mirror encoder modules are concatenated to the input to the decoder modules in order to aid the upsampling procedure.
Finally, a sigmoid convolution with filter size \num{1} is applied in order to produce the final segmentation probabilities.
This model has been implemented using the declarative Keras API in Tensorflow v2.1, yielding a final network with \num{7025329} trainable parameters.

\begin{figure}
  \centering
  \includegraphics[width=\textwidth]{Unet_ushape}
  \appcaption{%
    U-Net model architecture.
  }{%
    The vertical axis denotes the resolution of the features, resulting in the U-shape of U-Net.
    Figure has been generated by modifying a \texttt{tikz} example provided in the MIT licenced \texttt{PlotNeuralNet} library available at this URL:\@
    \protect\url{https://github.com/HarisIqbal88/PlotNeuralNet}.
  }%
  \label{fig:unet2}
\end{figure}
