Dropout is a regularization technique for neural networks intended to prevent \enquote{complex co-adaption of feature detectors}~\cite{dropout-original-paper}.
In practice this is achieved by randomly omitting hidden nodes from the neural network during each training step; effectively forcing hidden nodes to become less interdependent.
An alternative interpretation of the dropout procedure is that it is a computationally efficient form of model averaging, each dropout permutation being a model instance.
This technique has been empirically shown to significantly increase the test performance in several different settings.

Although originally intended for use in feedforward neural networks, dropout has been extensively applied in CNN architectures as well~\cite{dropout-cnn}.
Since there are no \enquote{nodes} to be omitted in fully convolutional layers, the dropout procedure needs to be adapted in order to be applicable in a CNN setting.
One approach is to introduce a randomly located square mask (\textit{cutout}) in the input image~\cite{dropout-cutout}.
Alternatively, \textit{stochastic depth dropout} randomly selects entire layers to be dropped, replacing them with identity functions instead~\cite{dropout-stochastic-depth}.
Dropout can also be integrated into max pooling layers, ignoring values at random during the search for the maximum value in the receptive field~\cite{max-pooling-dropout}.
This has become known as \textit{max-pooling dropout} and is illustrated in \cref{fig:max-pooling-dropout}.

\begin{figure}[H]
  %%%%%%%%%%%%%%%%%%% Local functions %%%%%%%%%%%%%%%%%%%
%% -- Draw marks
\newbox\dumbox% chktex 1
\newcommand{\mymark}[2]{%
  \setbox\dumbox=\hbox{#2}%
  \hbox to \wd\dumbox{\hss% chktex 1
    \tikz[overlay,remember picture,baseline=(#1.base)]{\node (#1) {\box\dumbox};}% chktex 36
    \hss}%
}
% Used to indicate dropout array indices
\newcommand{\dropout}[1]{{\setlength{\fboxsep}{0pt}\fcolorbox{black}{black}{#1}}}

%%%%%%%%%%%%%%%%%%% Local functions %%%%%%%%%%%%%%%%%%%

\begin{align*}
  \left[\begin{array}{cccc}
    1 & 8 & \mymark{oldTL1}{5} & \mymark{oldTR1}{0} \\
    8 & 11 & \mymark{oldBL1}{5} & \mymark{oldBR1}{4} \\
    8 & 17 &               10 & 11               \\
    9 & \mymark{old}{12} & 10 & 7 \\
  \end{array}\right]
  \hspace{0.5em}
  \begin{array}{ccc}
      \mymark{oldTL2}{\phantom{1}} & \phantom{1} & \mymark{oldTR2}{\phantom{1}}\\
      \phantom{1}  & \mymark{oldmycenter}{\phantom{1}} &              \phantom{0} \\
      \mymark{oldBL2}{\phantom{1}} & \phantom{0} & \mymark{oldBR2}{\phantom{0}}
  \end{array}
  =
  \left[\begin{array}{cccccc}
    11 & \mymark{oldC}{5} \\
    17 & 11 \\
  \end{array}\right]
  \\[2.25em]
  \left[\begin{array}{cccc}
    1 & \mymark{new}{8} & \mymark{TL1}{\dropout{5}} & \mymark{TR1}{0} \\
    \dropout{8} & 11 & \mymark{BL1}{\dropout{5}} & \mymark{BR1}{4} \\
    8 & \dropout{1} &               10 & 11               \\
    \dropout{9} & 12 & 10 & 7 \\
  \end{array}\right]
  \hspace{0.5em}
  \begin{array}{ccc}
      \mymark{TL2}{\phantom{1}} & \phantom{1} & \mymark{TR2}{\phantom{1}}\\
      \phantom{1}  & \mymark{mycenter}{\phantom{1}} &              \phantom{0} \\
      \mymark{BL2}{\phantom{1}} & \phantom{0} & \mymark{BR2}{\phantom{0}}
  \end{array}
  =
  \left[\begin{array}{cccccc}
    11 & \mymark{C}{4} \\
    12 & 11 \\
  \end{array}\right]
\end{align*}

\begin{tikzpicture}[overlay, remember picture,
    myedge1/.style={thin, opacity=.3, blue},
    myedge2/.style={thin, opacity=.3, green!40!black}]

  %% Draw boxes
  \draw[orange, fill=orange, fill opacity=.1]   (TL1.north west) rectangle (BR1.south east);
  \draw[orange, fill=orange, fill opacity=.1]   (oldTL1.north west) rectangle (oldBR1.south east);

  \draw[blue, fill=blue, fill opacity=.1] (TL2.north west) rectangle (BR2.south east)
    node[midway, opacity=1, color=black] {\Large $\max$};
  \draw[blue, fill=blue, fill opacity=.1] (oldTL2.north west) rectangle (oldBR2.south east)
    node[midway, opacity=1, color=black] {\Large $\max$};

  \draw[green!60!black, fill=green, fill opacity=.1] (C.north west) rectangle (C.south east);
  \draw[green!60!black, fill=green, fill opacity=.1] (oldC.north west) rectangle (oldC.south east);

  %% Draw blue lines
  \draw[myedge1] (TL1.north west) -- (TL2.north west);
  \draw[myedge1] (BL1.south west) -- (BL2.south west);
  \draw[myedge1] (TR1.north east) -- (TR2.north east);
  \draw[myedge1] (BR1.south east) -- (BR2.south east);
  \draw[myedge1] (oldTL1.north west) -- (oldTL2.north west);
  \draw[myedge1] (oldBL1.south west) -- (oldBL2.south west);
  \draw[myedge1] (oldTR1.north east) -- (oldTR2.north east);
  \draw[myedge1] (oldBR1.south east) -- (oldBR2.south east);

  %% Draw green lines
  \draw[myedge2] (TL2.north west) -- (C.north west);
  \draw[myedge2] (BL2.south west) -- (C.south west);
  \draw[myedge2] (TR2.north east) -- (C.north east);
  \draw[myedge2] (BR2.south east) -- (C.south east);
  \draw[myedge2] (oldTL2.north west) -- (oldC.north west);
  \draw[myedge2] (oldBL2.south west) -- (oldC.south west);
  \draw[myedge2] (oldTR2.north east) -- (oldC.north east);
  \draw[myedge2] (oldBR2.south east) -- (oldC.south east);

  % Draw arrow from old to new activation matrix
  \draw[->, line width=0.5mm, shorten >= 2mm] (old) ++ (-0.1mm, -4mm) -- node[auto, yshift=1mm] {Dropout} (new.north);
\end{tikzpicture}

  \appcaption{%
    An example application of \textit{max-pooling dropout} using a receptive field and stride of size $2 \times 2$.
  }{%
    A dropout probability of $p = 0.25$ has been used.
    Dropped values are shown as black boxes.
  }%
  \label{fig:max-pooling-dropout}
\end{figure}
