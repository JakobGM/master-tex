As the name implies, a central concept of convolutional neural networks is the \textit{convolution operator}.
Let the \textit{kernel}, $w$, be a $H_k \times W_k$ real matrix, and denote the activation of the previous layer at position $(x, y)$ as $a_{x, y}$.
The \textit{convolution operator}, $\circledast$, is then defined as
%
\begin{align*}
  w \circledast a_{x, y} = \sum_{i} \sum_{j} w_{i, j} ~ a_{x - i, y - j},
  \hspace{3em}
    a_{x, y} \in \mathbb{R},~
    w \in \mathbb{R}^{H_k \times W_k},
\end{align*}
%
where $(i, j)$ spans the index set of the kernel.
The region around $a_{x,y}$ which is involved in the convolution is referred to as the \textit{receptive field}.
We can generate a \textit{filtered image} by moving this receptive field over the entire input image.
The step size used when moving the receptive field is referred to as the \textit{stride size} of the convolution.
Such a \textit{moving convolution} is illustrated in \cref{fig:convolution}.

\begin{figure}[htb]
  %%%%%%%%%%%%%%%%%%% Local functions %%%%%%%%%%%%%%%%%%%
%% -- Draw marks
\newbox\dumbox% chktex 1
\newcommand{\mymark}[2]{%
  \setbox\dumbox=\hbox{#2}%
  \hbox to \wd\dumbox{\hss% chktex 1
    \tikz[overlay,remember picture,baseline=(#1.base)]{ \node (#1) {\box\dumbox}; }% chktex 36
    \hss}%
}

%%%%%%%%%%%%%%%%%%% Local functions %%%%%%%%%%%%%%%%%%%

\[
\underbracket[0.6pt][7pt]{
\left[\begin{array}{ccccccc}
  \mymark{TL3}{0} & 0 & 0 & 0 & 0 & \mymark{TR3}{0} \\
  0 & 1 & \mymark{TL1}{2} & 0 & \mymark{TR1}{0} & 0 \\
  0 & 5 & 3 &               0 & 4 &               0 \\
  0 & 0 & \mymark{BL1}{0} & 0 & \mymark{BR1}{7} & 0 \\
  0 & 9 & 3 & 0 & 0 & 0 \\
  \mymark{BL3}{0} & 0 & 0 & 0 & 0 & \mymark{BR3}{0} \\
\end{array}\right]
}_{\text{Zero-padded input}}
*
\underbracket[0.6pt][7pt]{
\left[\begin{array}{ccc}
  \mymark{TL2}{1} & 1 & \mymark{TR2}{1}\\
  1  & 1 &              0 \\
  \mymark{BL2}{1} & 0 & \mymark{BR2}{0}
\end{array}\right]
}_{\text{Kernel}}
=
\underbracket[0.6pt][7pt]{
\left[\begin{array}{ccccccc}
  1 & 8 & 5 & 0 \\
  8 & 11 & \mymark{C}{5} & 4 \\
  8 & 17 & 10 & 11 \\
  9 & 12 & 10 & 7 \\
\end{array}\right]
}_{\text{Convolved output}}
\]

\begin{tikzpicture}[overlay, remember picture,
    myedge1/.style={thin, opacity=.3, blue},
    myedge2/.style={thin, opacity=.3, green!40!black}]

  %% Draw boxes
  \draw[orange, fill=orange, fill opacity=.1]   (TL1.north west) rectangle (BR1.south east);
  \draw[blue, fill=blue, fill opacity=.1] (TL2.north west) rectangle (BR2.south east);
  \draw[green!60!black, fill=green, fill opacity=.1] (C.north west) rectangle (C.south east);

  %% Draw blue lines
  \draw[myedge1] (TL1.north west) -- (TL2.north west);
  \draw[myedge1] (BL1.south west) -- (BL2.south west);
  \draw[myedge1] (TR1.north east) -- (TR2.north east);
  \draw[myedge1] (BR1.south east) -- (BR2.south east);

  %% Draw green lines
  \draw[myedge2] (TL2.north west) -- (C.north west);
  \draw[myedge2] (BL2.south west) -- (C.south west);
  \draw[myedge2] (TR2.north east) -- (C.north east);
  \draw[myedge2] (BR2.south east) -- (C.south east);

  \fill[gray, fill opacity = 0.1, even odd rule]
    (TL3.north west) rectangle (BR3.south east)
    (TL3.south east) rectangle (BR3.north west);
\end{tikzpicture}

  \caption{
    Visualization of a kernel convolution with a $3 \times 3$ kernel over an image of size $4 \times 4$ with additional zero-padding and stride size of $1 \times 1$.
    The \textit{receptive field} is shown in \textcolor{orange}{orange}, the respective kernel weights in \textcolor{blue}{blue}, and the resulting convolution output in \textcolor{green}{green}.
    The zero padding of the input image is shown in gray.
  }%
  \label{fig:convolution}
\end{figure}

In the case of input images or activations comprised of more than one channel, independent two-dimensional kernels are constructed for each channel and the convolved outputs are finally summed in order to attain a single feature map.
The concept of a \textit{kernel} predates neural networks as it has been used for feature extraction in the field of image processing for many years~\cite[p.~11]{computer_vision_history}.
The kernel weights determine the type of features being extracted from the given input image, some common interpretable kernels are given below.

\begin{footnotesize}\begin{align*}
  \underbracket[0.6pt][7pt]{
    w_1 =
    \begin{bmatrix}
      0 & 0 & 0 \\
      0 & 1 & 0 \\
      0 & 0 & 0 \\
    \end{bmatrix}
  }_{\text{Identity kernel}},
  &&
  \underbracket[0.6pt][7pt]{
    w_2 =
    \begin{bmatrix}
      -1 & -1 & -1 \\
      -1 & 8 & -1 \\
      -1 & -1 & -1 \\
    \end{bmatrix}
  }_{\text{Edge detection kernel}},
  &&
  \underbracket[0.6pt][7pt]{
    w_3 =
    \frac{1}{9}
    \begin{bmatrix}
      1 & 1 & 1 \\
      1 & 1 & 1 \\
      1 & 1 & 1 \\
    \end{bmatrix}
  }_{\text{Normalized box blur kernel}},
  &&
  \underbracket[0.6pt][7pt]{
    w_4 =
    \frac{1}{16}
    \begin{bmatrix}
      1 & 2 & 1 \\
      2 & 4 & 2 \\
      1 & 2 & 1 \\
    \end{bmatrix}
  }_{\text{Gaussian blur kernel}}.
\end{align*}\end{footnotesize}

It is important to notice that kernel convolution has the additional effect of reducing the dimensionality of the input image.
Firstly, pixels along the image border are partially ignored since the receptive field can not be properly centered on these pixel.
Secondly, a horizontal stride of $W_k > 1$ or a vertical stride of $H_k > 1$ will cause additional dimensional reduction.
For an image of size $H \times W$ and a kernel of size $H_k \times W_k$, the input image is reduced to size
%
\begin{equation*}
  \floor{(H - H_k + H_s) / H_s}
  \times
  \floor{(W - W_k + W_s) / W_s}.
\end{equation*}
%
as shown by~\cite{dive-into-deep-learning}.
The reduction in dimensionality when using stride sizes of one is often undesirable, and for this reason it is common to add a \textit{padding} filled with zero-values along the edges of the input image.
Applying a padding of height $H_p$ at the horizontal borders and a padding of width $W_p$ at the vertical borders results in a feature map of size
%
\begin{equation*}
  \floor{(H - H_k + H_s + \mathbf{H_p}) / H_s}
  \times
  \floor{(W - W_k + W_s + \mathbf{W_p}) / W_s}.
\end{equation*}
%
If we assume the input height and width to be divisible by the stride height and width respectively, we can set $H_p = H_k - 1$ and $W_p = W_k - 1$ in order to attain an output shape of $(H / H_s) \times (W / W_s)$~\cite{dive-into-deep-learning}.
Such a padding is shown in gray in \cref{fig:convolution}.

CNNs apply multiple different convolutions to the same input, resulting in a set of differently filtered outputs.
After having applied the layer's activation function to the output (see upcoming section about \enquote{activation functions}) and the activations have been downsampled (see upcoming \enquote{pooling} section), the filtered outputs are passed onto the next layer.
The number of filters are usually increased as you move deeper into the network where the resolution has been increasingly downsampled.
Unlike classical image processing, where kernel weights are carefully selected in order to construct an intended type of feature extraction, CNNs let each kernel weight be a trainable parameter.
As the network is trained each kernel learns to extract features which are of use for the subsequent layers.

An important aspect of convolution is that the kernel weights remain unchanged as the receptive field is moved over the input image.
This \textit{parameter sharing} results in regions being treated identically no matter where in the image they are situated~\cite{visint-cnn}.
The sharing of parameters has the benefit of reducing the parametric complexity of the network, thus decreasing the computational cost of training it.
Finally, compared to a more classical \textit{fully connected feedforward network}, which operates over flattened vectors, a fully convolutional neural network operates over images in matrix form, thus taking the spatial relationship between pixels into account.
