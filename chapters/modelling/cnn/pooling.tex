The last layer in a given CNN block is conventionally a \textit{downsampling} operation, most often referred to as a \textit{pooling} layer.
As with convolution, this operation has biological influences as it is inspired by a model of the mammalian visual cortex~\cite[p.~966]{visint-cnn}.
The reduction in spatial resolution is considered to be one of the main reasons for why CNNs portray a high degree of translational and rotational invariance~\cite{cnn-translational-invariance}.
As with moving convolution, pooling is implemented by moving a receptive field of size greater than $1$, typically $2 \times 2$, over the activations and mapping these values into a lower dimensional space.
There are several different ways to define such a mapping, the two most common being \textit{max pooling} and \textit{average pooling}, which respectively retrieve the maximum value and average value from the receptive field.
The former is exemplified in \cref{fig:max-pooling}.

\begin{figure}[H]
  %%%%%%%%%%%%%%%%%%% Local functions %%%%%%%%%%%%%%%%%%%
%% -- Draw marks
\newbox\dumbox% chktex 1
\newcommand{\mymark}[2]{%
  \setbox\dumbox=\hbox{#2}%
  \hbox to \wd\dumbox{\hss% chktex 1
    \tikz[overlay,remember picture,baseline=(#1.base)]{\node (#1) {\box\dumbox};}% chktex 36
    \hss}%
}

%%%%%%%%%%%%%%%%%%% Local functions %%%%%%%%%%%%%%%%%%%

\[
\underbracket[0.6pt][7pt]{
\left[\begin{array}{cccc}
  1 & 8 & \mymark{TL1}{5} & \mymark{TR1}{0} \\
  8 & 11 & \mymark{BL1}{5} & \mymark{BR1}{4} \\
  8 & 17 &               10 & 11               \\
  9 & 12 & 10 & 7 \\
\end{array}\right]
}_{\text{Activations}}
\hspace{0.5em}
\underbracket[0.6pt][7pt]{
\begin{array}{ccc}
    \mymark{TL2}{\phantom{1}} & \phantom{1} & \mymark{TR2}{\phantom{1}}\\
    \phantom{1}  & \mymark{mycenter}{\phantom{1}} &              \phantom{0} \\
    \mymark{BL2}{\phantom{1}} & \phantom{0} & \mymark{BR2}{\phantom{0}}
\end{array}
}_{\text{Pool operation}}
=
\underbracket[0.6pt][7pt]{
\left[\begin{array}{cccccc}
  11 & \mymark{C}{5} \\
  17 & 11 \\
\end{array}\right]
}_{\text{Pooled output}}
\]

\begin{tikzpicture}[overlay, remember picture,
    myedge1/.style={thin, opacity=.3, blue},
    myedge2/.style={thin, opacity=.3, green!40!black}]

  %% Draw boxes
  \draw[orange, fill=orange, fill opacity=.1]   (TL1.north west) rectangle (BR1.south east);
  \draw[blue, fill=blue, fill opacity=.1] (TL2.north west) rectangle (BR2.south east)
    node[midway, opacity=1, color=black] {\Large $\max$};
  \draw[green!60!black, fill=green, fill opacity=.1] (C.north west) rectangle (C.south east);

  %% Draw blue lines
  \draw[myedge1] (TL1.north west) -- (TL2.north west);
  \draw[myedge1] (BL1.south west) -- (BL2.south west);
  \draw[myedge1] (TR1.north east) -- (TR2.north east);
  \draw[myedge1] (BR1.south east) -- (BR2.south east);

  %% Draw green lines
  \draw[myedge2] (TL2.north west) -- (C.north west);
  \draw[myedge2] (BL2.south west) -- (C.south west);
  \draw[myedge2] (TR2.north east) -- (C.north east);
  \draw[myedge2] (BR2.south east) -- (C.south east);
\end{tikzpicture}

  \caption{%
    Example of a \textit{max-pooling} operation with a receptive field of size $2 \times 2$ and an identical stride size.
    The receptive field is shown in \textcolor{orange}{orange} and the respective pooled output is shown in \textcolor{green}{green}.
  }%
  \label{fig:max-pooling}
\end{figure}

As can be seen in \cref{fig:max-pooling}, using a receptive field and stride of size $2 \times 2$ will yield a downsampled image with one quarter as many pixels as the original input.
