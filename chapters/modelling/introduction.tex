The field of \textit{computer vision} got started in the early 1970s~\cite[p.~10]{computer_vision_history}, where one of the problems being tackled is the three-dimensional reconstruction of a scene from two-dimensional data.
Most of the early research in the field revolved around manually designed feature extraction and processing techniques, but statistical techniques started to become popular in the 1990s~\cite[p.~15]{computer_vision_history}.
The statistical approach eventually morphed into the field of \textit{machine learning}, where most of the research advances are made today~\cite[p.~17]{computer_vision_history}.

We have already provided an overview of the problem domains of interest in \cref{sec:segmentation-description}, but we will now provide a more in-depth description of all the relevant theoretical concepts.
\textit{Convolutional neural networks} (CNNs) have been applied to image segmentation problems with great success~\cite[p.~1]{image_recognition}, and \cref{sec:cnn} provides a theoretic overview of the elementary building blocks used to construct modern CNN architectures.
\Cref{sec:semantic-segmentation} will go more in-depth into \emph{image segmentation} specifically, treating topics such as popular evaluation metrics, optimization losses, and the current state-of-the-art.
The section is finalized by a description of the CNN architecture used in this work, namely the U-Net architecture.
Having explained the relevant concepts of semantic segmentation, \cref{sec:surface-raster-model} transitions to the problem of predicting rasterized surface normal vectors.
The section starts by describing previous related work, followed by a description of a modified U-Net architecture for predicting such vectors.
Finally, suitable loss functions for surface normal vector predictions are discussed.
\cref{sec:optimization,sec:raster-normalization} end this chapter by describing the general process of training CNN networks and how the remote sensing raster data should be normalized before being passed into the neural networks.
