\chapter{Introduction}%
\label{sec:introduction}
\pagenumbering{arabic}

\textit{Remote sensing} is the process of gathering information about an object without making physical contact, one such technology being \textit{aerial photography}.
Although aerial RGB photography is intuitively interpretable for humans, it is fundamentally two-dimensional.
\textit{LiDAR}, another remote sensing technology, is able to measure distances to object surfaces by directing a beam of light and measuring the time of arrival and wavelength of the ensuing reflection.
The resulting data can therefore be used to construct a three-dimensional spatial representation of the object of interest.
LiDAR has been applied in a wide array of fields such as meteorology~\cite{lidar_meteorology_1966}, forestry analysis~\cite{lidar_forestry_2000}, urban flood modelling~\cite{lidar_flood_2013}, and autonomous driving systems~\cite{lidar_self_driving_2018}.

One of the applications of LiDAR technology is the construction of \textit{digital surface models} (DSMs).
DSMs are grayscale images representing the earth's surface including all above-surface objects such as natural canopy and human-made objects.
In contrast, \textit{digital terrain models} (DTMs) represent the elevation of the \textit{bare} ground where all above-surface objects have been artificially removed.
While DTMs are often used in geographic and cartographic applications, DSMs can be used for localization and classification of objects above ground.

LiDAR data and aerial photography is usually provided by the respective cadastral authority in a given country.
Cadastral authorities are also responsible for keeping records of cadastral data such as cadastral plots, roads, and buildings.
The exact type and quality of this data varies substantially between countries and sometimes even between administrative regions in the same country.
This raises the question: \enquote{Can high-fidelity insights be inferred from otherwise low-fidelity geographic data?}.
\Cref{fig:data-enchancement} shows an outline of the possible \enquote{data enhancements} which are of interest within this domain. %chktex 2

\begin{figure}
  \includegraphics[width=\linewidth]{data-enchancement}
  \appcaption{%
    Classification of geographic data quality.
  }{%
    The classifications reflect a general observed trend in data sets, and a given region may therefore not fit into exactly one of these categories.
    Some of the data types mentioned here will be described in \cref{sec:data}.
  }%
  \label{fig:data-enchancement}
\end{figure}

The \textit{Norwegian Mapping and Cadastre Authority} (\textit{Statens Kartverk}) provides geographic data of uniquely high quality for the entirety of Norway.
This offers an opportunity to train supervised machine learning models on lower fidelity data in order to infer higher fidelity features.
Such models can then be applied in other regions where only low-fidelity data is available as a method of data enhancement.

The goal of my specialization project~\cite{specialization-project} from \citedate{specialization-project} was to infer two-\linebreak{}dimensional \textit{building outlines} from aerial photography and LiDAR elevation measurements.
A \textit{building outline} is a two-dimensional representation of building \enquote{footprint}.
Such data can be used for map annotations, flood risk analysis, and population density estimates, amongst other applications.
The identification of building outlines from remote sensing data is considered to be medium-fidelity target inference by using low-fidelity features, and can be formulated as a so-called \textit{semantic segmentation} task\footnote{The concepts \enquote{semantic segmentation} and \enquote{instance segmentation} will be formally defined in \cref{sec:segmentation-description}.}.

This master's thesis concerns itself with the reconstruction of three-dimensional \textit{roof surface polygons} from remote sensing data.
Roof surface polygons are completely flat geometries, which when combined form the spatial shape of entire roof structures.
The detection of roof surface polygons is formulated as an \textit{instance segmentation} problem, a task which produces high-fidelity targets.
Three-dimensional representations of roof structures can for example be used for urban planning purposes.
Another application, which incidentally prompted my interest in this topic, is the use of roof surface geometries to estimate the potential energy production of roof-mounted solar panel installations.

The topic of this master's thesis is a natural extension of much of the work already presented in my specialization project.
A sequential two-step method has been developed in order to detect roof surface polygons,
\begin{enumerate}[noitemsep]
  \item Determine which pixels that contain roof structures (semantic segmentation).
  \item Assign a specific roof surface to each \enquote{roof pixel} (instance segmentation).
\end{enumerate}
The first step is essentially a minor reformulation of the task already solved in my specialization project.
For this reason, much of the theory, general methods, and specific source code from my specialization project has been incorporated into this work.

\subsection*{Research questions}

Geographic data, such as building outlines and roof surfaces, are formatted in an unsuitable way for direct machine learning, and must therefore be purposefully transformed and pre-processed.
The development of a data pipeline for geographic data is the first topic of research in this thesis.
%
\begin{description}
  \item[RQ1] How can geographic data representations be transformed into suitable formats for machine learning?
\end{description}
%
After having developed such a pipeline, the focus will be to develop an instance segmentation model for identifying roof surfaces with \emph{raster} data from this pipeline.
The use of aerial photography and LiDAR data from the Norwegian municipality of Trondheim will be investigated, as well as the combination of these two data sources.
%
\begin{description}
  \item[RQ2] How can aerial photography and/or LiDAR data be used in order to infer accurate roof surface instance segmentation maps?
\end{description}
%
The resulting instance segmentation map, which represents rasterized roof surfaces in two dimensions, should be \emph{vectorized}.
That is, the machine learning model's raster predictions should be converted to three-dimensional vector polygons, a data format which is more usable for the most common applications of roof surface geometries.
%
\begin{description}
  \item[RQ3] How can three-dimensional roof surface \emph{polygons} be produced from predicted instance segmentation \emph{raster} maps.
\end{description}
%
Answering this last research question will require the development of additional post-processing methods.

\subsection*{Thesis disposition}

We will start by providing an introduction to the world of \textit{Geographic Information Systems} (GIS); the field which concerns itself with representing geographic data, in \cref{sec:data}.
We will also describe how to pre-process such geographic data in order to produce rasters which are suitable for training accurate machine learning models.
An overview of the problem domain of image segmentation and the methods currently being applied in the field will be provided in \cref{sec:modeling}, a chapter which will also describe the specific model architectures which consume and produce the data formats described in the previous chapter.
The post-processing required in order to produce vectorized surface polygons from predicted rasters will be described in \cref{chap:post-processing}.
Finally, the training procedure and experimental results will be presented and discussed in \cref{sec:experiments}.

The chapter describing geospatial data structures and how they are pre-\linebreak{}processed (\cref{sec:data}) has been placed \emph{before} the chapter about model architectures and techniques related to image segmentation (\cref{sec:modeling}), since the choice of model architecture is highly dependent on how the ground truth target rasters have been constructed.
In order to provide some additional context before describing the pre-processing of our geospatial datasets, we start by giving a formal overview of the problem domains involved in this thesis.
This upcoming section should strictly belong to \cref{sec:modeling}, but the information is still considered essential in order to understand what motivates the decisions made during the pre-processing of the original raw geospatial data.

\section*{Problem description}%
\label{sec:segmentation-description}
For any given image we can pose three relevant \textit{image recognition} questions~\cite{image_recognition}:
%
\begin{enumerate}[]
  \item \textbf{Identification:} Does the image contain any object of interest?
  \item \textbf{Localization:} Where in the image are the objects situated?
  \item \textbf{Classification:} To which categories do the objects belong to?
\end{enumerate}
%
We will concern ourselves with only one object category (class) at any time, that class being roof surfaces, and will simplify the upcoming theory accordingly with this simplification in mind.
The localization and classification of objects in a given image can be performed at different granularity levels, as shown by the \emph{columns} in \cref{fig:segmentation-types}.
The \emph{rows} of \cref{fig:segmentation-types} show how the specific definition of what exactly constitutes an object influences the problem to be solved, where the top row considers entire building to be single objects, while the bottom row considers each individual roof surface to be distinct objects.
It is the latter definition which is of interest in this work.

\begin{figure}[t]
  \includegraphics[width=\linewidth]{segmentation-types}
  \includegraphics[width=\linewidth]{surface-segmentation-types}
  \appcaption{%
    Different granularities for single-class construction localization, using the Trondheim 2017 data set.
  }{%
    Bounding box regression is shown on the left, semantic segmentation in the middle, and instance segmentation on the right.
    The top row defines entire buildings as the objects of interest, while the bottom row considers each individual roof surface as distinct objects.
  }%
  \label{fig:segmentation-types}
\end{figure}

\textit{Bounding box regression} concerns itself with finding the smallest possible rectangles which envelopes the objects of interest.
The sides of the rectangles may either by oriented parallel to the axis directions, or rotated in order to attain the smallest possible envelope.
The bounding box will therefore necessarily contain pixels that are not part of the object itself whenever the object shape is not perfectly rectangular.

\textit{Semantic segmentation} rectifies this issue by classifying each pixel in the image independently, i.e. \textit{pixel-wise} classification, producing a so-called classification \textit{mask}.
\textit{Instance segmentation} distinguishes between pixels belonging to different objects of the same class, while \textit{semantic segmentation} does not make this distinction.
Since a bounding box can be directly derived from a semantic segmentation mask, and a semantic segmentation mask can be directly derived from instance segmentation mask; the problem complexity of these tasks are as follows:
%
\begin{equation*}
  \text{Bounding box regression}
  <
  \text{Semantic segmentation}
  <
  \text{Instance segmentation}.
\end{equation*}
%
An image of width $W$ and height $H$ consisting of $C$ channels is represented by a $W \times H \times C$ tensor, $X \in \mathbb{R}^{W \times H \times C}$.
This is somewhat simplified, but we will give a more nuanced description in \cref{sec:raster-data}.
Single-class semantic segmentation can therefore be formalized as constructing a binary predictor $\tilde{f}$ of the form:
%
\begin{equation*}
  \tilde{f}: \mathbb{R}^{W \times H \times C} \rightarrow \mathbb{B}^{W \times H}, \hspace{2em} \mathbb{B} \defeq \{0, 1\}.
\end{equation*}
%
Where $\mathbb{B}^{W \times H}$ denotes a boolean matrix, $1$ indicating that the pixel is part of the object class of interest, and $0$ indicates the opposite.
In practice, however, statistical models will often predict a pixel-wise class \textit{confidence} in the continuous domain $[0, 1]$,
%
\begin{equation*}
  \hat{f}: \mathbb{R}^{W \times H \times C} \rightarrow {[0, 1]}^{W \times H},
\end{equation*}
%
but a binary predictor can be easily constructed by choosing a suitable threshold, $T$, for which to distinguish positive predictions from negative ones
%
\begin{equation*}
  \tilde{f}(X) = \hat{f}(X) > T, \hspace{2em} X \in \mathbb{R}^{W \times H \times C}.
\end{equation*}
%
The choice of the threshold value $T$ will affect the resulting \textit{sensitivity} and \textit{specificity} metrics of the model predictions, metrics which will be explained in \cref{sec:semantic-segmentation}.

When performing single-class \emph{instance} segmentation, a binary prediction mask is not sufficiently expressive.
Assuming that no more than $m$ individual instances can be simultaneously represented by any given input tensor X, the task is to assign an instance label $l \in \{0, 1, 2, \dots, m\}$ to every single pixel in the input tensor.
The assignment of $l = 0$ means that no object overlaps the given pixel.
An instance predictor $\hat{f}_{\predinstraster}$ is therefore a function producing a label array $\predinstraster$ of the form
\begin{equation*}
  \hat{f}_{\predinstraster}: \mathbb{R}^{W \times H \times C}
  \rightarrow
  \left\{
    0, 1, 2, \dots, m
  \right\}^{W \times H \times C}.
\end{equation*}

Instead of representing each predicted instance as a set of \emph{raster} pixels which share the same label value $l$ in $\predinstraster$, they can be represented as two-dimensional \emph{vectorized} polygons which enclose each given instance instead (\textit{polygon segmentation}).
The specific data format used in order to represent vectorized polygons is presented in \cref{sec:vector-data}.
For the case of three-dimensional objects, the instance polygons can be constructed to be three-dimensional as well, sometimes referred to as \textit{surface segmentation}~\cite{surface-segmentation-1,surface-segmentation-2}.
Since the three-dimensional polygons in a surface segmentation can be projected into a two-dimensional plane in order to produce a polygon segmentation map, and since two-dimensional polygons can be \textit{rasterized} in order to create an instance segmentation map, the problem complexity of these tasks are as follows:
%
\begin{equation*}
  \text{Instance segmentation}
  <
  \text{Polygon segmentation}
  <
  \text{Surface segmentation}.
\end{equation*}

This thesis presents a machine learning pipeline which produces vectorized surface polygon segmentations from remote sensing data.
The pipeline does this by first predicting a semantic segmentation map, which is then partitioned into an instance segmentation map.
This instance segmentation map is then vectorized in order to produce a polygon segmentation map.
Finally, the three dimensional elevation and orientation of each polygon is inferred from the original LiDAR data in order to produce a surface segmentation map.

