\textit{Remote sensing} is the process of gathering information about an object without making physical contact, one such technology being \textit{aerial photography}.
Although aerial RGB photography is intuitively interpretable for humans, it is fundamentally two-dimensional.
\textit{LiDAR}, another remote sensing technology, is able to measure distances to object surfaces by directing a beam of light and measuring the time of arrival and wavelength of the ensuing reflection.
The resulting data can therefore be used to construct a three-dimensional spatial representation of the object of interest.
LiDAR has been applied in a wide array of fields such as meteorology~\cite{lidar_meteorology_1966}, forestry analysis~\cite{lidar_forestry_2000}, urban flood modeling~\cite{lidar_flood_2013}, and autonomous driving systems~\cite{lidar_self_driving_2018}.

One of the applications of LiDAR technology is the construction of \textit{digital surface models} (DSMs).
DSMs are grayscale images representing the earth's surface including all above-surface objects such as natural canopy and human-made objects.
In contrast, \textit{digital terrain models} (DTMs) represent the elevation of the \textit{bare} ground where all above-surface objects have been artificially removed.
While DTMs are often used in geographic and cartographic applications, DSMs can be used for localization and classification of objects above ground.

LiDAR data and aerial photography is usually provided by the respective cadastral authority in a given country.
Cadastral authorities are also responsible for keeping records of cadastral data such as cadastral plots, roads, and buildings.
The exact type and quality of this data varies substantially between countries and sometimes even between administrative regions in the same country.
This raises the question: \enquote{Can high-fidelity insights be inferred from otherwise low-fidelity geographic data?}.
\Cref{fig:data-enchancement} shows an outline of the possible \enquote{data enhancements} which are of interest within this domain. %chktex 2

\begin{figure}
  \includegraphics[width=\linewidth]{data-enchancement}
  \appcaption{%
    Classification of geographic data quality.
  }{%
    The classifications reflect a general observed trend in data sets, and a given region may therefore not fit into exactly one of these categories.
    Some of the data types mentioned here will be described in \cref{sec:data}.
  }%
  \label{fig:data-enchancement}
\end{figure}

The \textit{Norwegian Mapping and Cadastre Authority} (\textit{Statens Kartverk}) provides geographic data of uniquely high quality for the entirety of Norway.
This offers an opportunity to train supervised machine learning models on lower fidelity data in order to infer higher fidelity features.
Such models can then be applied in other regions where only low-fidelity data is available as a method of data enhancement.

The goal of my specialization project~\cite{specialization-project} from \citedate{specialization-project} was to infer two-\linebreak{}dimensional \textit{building outlines} from aerial photography and LiDAR elevation measurements.
A \textit{building outline} is a two-dimensional representation of building \enquote{footprint}.
Such data can be used for map annotations, flood risk analysis, and population density estimates, amongst other applications.
The identification of building outlines from remote sensing data is considered to be medium-fidelity target inference by using low-fidelity features, and can be formulated as a so-called \textit{semantic segmentation} task\footnote{The concepts \enquote{semantic segmentation} and \enquote{instance segmentation} will be formally defined in \cref{sec:segmentation-description}.}.

This master's thesis concerns itself with the reconstruction of three-dimensional \textit{roof surface polygons} from remote sensing data.
Roof surface polygons are completely flat geometries, which when combined form the spatial shape of entire roof structures.
The detection of roof surface polygons is formulated as an \textit{instance segmentation} problem, a task which produces high-fidelity targets.
Three-dimensional representations of roof structures can for example be used for urban planning purposes.
Another application, which incidentally prompted my interest in this topic, is the use of roof surface geometries to estimate the potential energy production of roof-mounted solar panel installations.

The topic of this master's thesis is a natural extension of much of the work already presented in my specialization project.
A sequential two-step method has been developed in order to detect roof surface polygons,
\begin{enumerate}[noitemsep]
  \item Determine which pixels that contain roof structures (semantic segmentation).
  \item Assign a specific roof surface to each \enquote{roof pixel} (instance segmentation).
\end{enumerate}
The first step is essentially a minor reformulation of the task already solved in my specialization project.
For this reason, much of the theory, general methods, and specific source code from my specialization project has been incorporated into this work.

