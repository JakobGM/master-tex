Geographic data, such as building outlines and roof surfaces, are formatted in an unsuitable way for direct machine learning, and must therefore be purposefully transformed and pre-processed.
The development of a data pipeline for geographic data is the first topic of research in this thesis.
%
\begin{description}
  \item[RQ1] How can geographic data representations be transformed into suitable formats for machine learning?
\end{description}
%
After having developed such a pipeline, the focus will be to develop an instance segmentation model for identifying roof surfaces with \emph{raster} data from this pipeline.
The use of aerial photography and LiDAR data from the Norwegian municipality of Trondheim will be investigated, as well as the combination of these two data sources.
%
\begin{description}
  \item[RQ2] How can aerial photography and/or LiDAR data be used in order to infer accurate roof surface instance segmentation maps?
\end{description}
%
The resulting instance segmentation map, which represents rasterized roof surfaces in two dimensions, should be \emph{vectorized}.
That is, the machine learning model's raster predictions should be converted to three-dimensional vector polygons, a data format which is more usable for the most common applications of roof surface geometries.
%
\begin{description}
  \item[RQ3] How can three-dimensional roof surface \emph{polygons} be produced from predicted instance segmentation \emph{raster} maps.
\end{description}
%
Answering this last research question will require the development of additional post-processing methods.

We will now formalize the problem domains which these research questions belong to, namely \textit{semantic segmentation}, \textit{instance segmentation}, \textit{polygon segmentation}, and \textit{surface segmentation}.
