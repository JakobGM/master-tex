Important that the algorithm does not presume the number of clusters.
First a conservative pass which determines number of clusters and identifies the pixels which are a "sure thing".
Then a second clustering pass which encodes our domain knowledge.

\begin{wrapfigure}{r}{0.35\textwidth}
  \begin{center}
    \includegraphics[width=0.35\textwidth]{tile-data/130-0/surface_normals_3+segmented_normals_2/polar_normals_true_labels.pdf}
  \end{center}
  \caption{TODO.}%
  \label{fig:polar-normal-true-labels}
\end{wrapfigure}
\lipsum[1-2]

\subsection{DBSCAN}

\begin{figure}[H]
  \centering
  \includegraphics[width=\textwidth]{cosine-similarity-distribution}
  \caption{All cosine similarities for all \numtesttiles~test tiles.}
\end{figure}

\begin{wrapfigure}{r}{0.35\textwidth}
  \begin{center}
    \includegraphics[width=0.35\textwidth]{tile-data/130-0/surface_normals_3+segmented_normals_2/polar_normals_dbscan_labels.pdf}
  \end{center}
  \caption{TODO.}%
  \label{fig:polar-normal-dbscan-labels}
\end{wrapfigure}
\lipsum[1-2]

\begin{figure}[H]
  \centering
  \includegraphics[]{dbscan.tikz}
  \caption{Application of DBSCAN on predicted surface normal raster.}
\end{figure}

\subsection{\texorpdfstring{$k$}{k}-nearest neighbour noise classification}

\begin{figure}[H]
  \centering
  \includegraphics[]{knn.tikz}
  \caption{Application of KNN on DBSCAN-identified noise.}
\end{figure}

\newpage
\begin{wrapfigure}{r}{0.35\textwidth}
  \begin{center}
    \includegraphics[width=0.35\textwidth]{tile-data/130-0/surface_normals_3+segmented_normals_2/polar_normals_knn_labels.pdf}
  \end{center}
  \caption{%
    TODO.
  }%
  \label{fig:polar-normal-knn-labels}
\end{wrapfigure}

DBSCAN accuracy = \num{9978233830845771}\%.
Noise accuracy = \num{76.86832740213523}\%.
Final accuracy = \num{97.94109236488419}\%.

\lipsum[1-2]
