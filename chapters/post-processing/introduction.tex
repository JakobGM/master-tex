Assume that we have two machine learning models; firstly $\segmodel$ which is able to predict \emph{semantic} segmentation masks $\predsegraster$, and secondly $\normmodel$ which is able to predict surface normal rasters $\prednormraster$.
These two models could conceivably be sub-models of a \emph{single} multi-task model, both accepting remote sensing data in the form of aerial photography and/or LiDAR measurements as illustrated in \cref{fig:multitask-prediction}.
\begin{figure}[H]
  \centering
  \includegraphics{multitask-prediction.tikz}
  \caption{%
    Demonstration of unprocessed output from a surface raster machine learning pipeline.
  }%
  \label{fig:multitask-prediction}
\end{figure}
\noindent
As discussed in \cref{sec:surface-rasterization} -- \enquote{\nameref{sec:surface-rasterization}}, the goal is to convert these two model predictions belonging to the surface \emph{raster} domain $R$ back to the \emph{vector} polygon domain $V$ by constructing a suitable pseudoinverse mapping $m^{\dagger}: R \rightarrow V$ as illustrated in \cref{fig:pseudoinverse}.
\begin{figure}[H]
  \centering
  \includegraphics{pseudoinverse.tikz}
  \caption{Pseudoinverse mapping, $m^{\dagger}$, mapping from the raster domain to the vector domain.}%
  \label{fig:pseudoinverse}
\end{figure}
\noindent
More specifically, we will construct a pseudoinverse mapping which accepts two inputs, a predicted surface normal raster and a semantic segmentation map, and produces three-dimensional vector polygons, which we will denote as $\predpolygons$,
\begin{equation*}
  \predpolygons
  \defeq
  \pseudoinverse\left(
    \predsegraster,
    \prednormraster
  \right).
\end{equation*}
This pseudoinverse mapping should produce three-dimensional polygons which can be considered close to the ground truth polygons $\polygons$, where the notion of \enquote{closeness} between $\predpolygons$ and $\polygons$ is expressed in a \textit{polygon distance metric} $d: V \times V \rightarrow \mathbb{R}$.
The main objective of the pseudoinverse mapping is therefore to minimize
\begin{equation*}
  d\left(
    \predpolygons,
    \polygons
  \right)
  =
  d\left(
    \pseudoinverse\left(\predsegraster, \prednormraster\right),
    \polygons
  \right).
\end{equation*}
The model predictions, $\predsegraster$ and $\prednormraster$, are not perfect, and the pseudoinverse should therefore be \textit{robust} relative to the types of errors commonly produced by the models, $\segmodel$ and $\normmodel$, producing minimal error as defined by the distance metric $d$.
Constructing a suitable pseudoinverse is therefore not only highly dependent on the surface raster format itself, but also the specific behaviour of the models used to produce the raster format.
In the evaluation of a pseudoinverse mapping it may still be beneficial to take the models themselves out of the equation by replacing the predicted rasters with the ground truth rasters $m(\polygons)$, thus verifying that the pseudoinverse mapping also minimizes
\begin{equation*}
  d\left(
    \pseudoinverse\left(m\left(\polygons\right)\right),
    \polygons
  \right).
\end{equation*}
A pseudoinverse mapping which performs badly under this \enquote{round-trip metric} will likely perform badly under the pseudoinverse mapping of model predictions as well.
\newpage

In the following sections we will present a pseudoinverse mapping which is considered highly suitable for producing roof surface polygons, a mapping which encodes our specific domain knowledge about the general geometric properties of roof surfaces.
Our main idea is to use the predicted surface normal raster in order to \textit{partition} the \textit{semantic} segmentation map into an \textit{instance} segmentation map, where the instance segmentation map can subsequently be used in order to construct enclosing vector polygons for each instance.
The three-dimensional orientation of each polygon can finally be inferred from the predicted surface normals in conjunction with the original LiDAR data.

Our implementation applies a divide-and-conquer approach to the problem by dividing the semantic segmentation map into mutually disjunct sub-regions and partitioning these sub-regions entirely independent of each other.
This first-pass partitioning is described in \cref{sec:connected-region-labeling} -- \enquote{\nameref{sec:connected-region-labeling}}.
A second-pass partitioning is performed by grouping together connected regions which share the same normal vector orientation.
The task of determining which pixels $(i, j)$ which share the same values for $N_{i, j}$ is formulated as a \textit{clustering problem}, the solution of which will be presented in \cref{sec:instance-clustering}.
A final conversion of instance segmentation maps to vector polygons will be described in \cref{sec:vectorization}.

Before delving into the details of the implementation, we begin by summarizing the entire post-processing algorithm in conceptual terms:

\begin{leftbar}
  \noindent
  Given a predicted segmentation map $\predsegraster$ and surface normal raster $\prednormraster$:
  \begin{itemize}[leftmargin=*]
    \item Threshold segmentation activations $\predsegraster$ according to some tolerance $\texttt{TOL}$ in order to construct \emph{binary} segmentation raster $\predbinarysegraster$.
    \item Identify connected sub-regions of the binary segmentation raster $\predbinarysegraster$, and sub-divide the following processing independently across these sub-regions.
    \item Apply a conservative clustering algorithm which does \emph{not} require the specification of the number of clusters \textit{a priori}.
    \item Apply a second clustering algorithm which encodes suitable domain knowledge in order to classify the remaining points, producing a labeled partition of the original binary segmentation raster.
    \item Construct two-dimensional vector polygons which enclose each instance region.
    \item Simplify these vector polygons.
    \item Reconstruct $z$-coordinates of each polygon vertex.
  \end{itemize}
\end{leftbar}
