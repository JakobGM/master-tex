Assume that we now have two separate models; $\segmodel$, which is able to predict \emph{semantic} segmentation masks, $\predsegraster$, and $\normmodel$, which is able to predict surface normal rasters, $\prednormraster$.
The main prediction pipeline has been demonstrated in \cref{fig:multitask-prediction}.
% $\segmodel$ and $\normmodel$ could conceivably be sub-models of a \emph{single} multi-task model, but that is besides the point.
\begin{figure}[H]
  \centering
  \includegraphics{multitask-prediction.tikz}
  \caption{%
    Demonstration of unprocessed output from a surface raster machine learning pipeline.
  }%
  \label{fig:multitask-prediction}
\end{figure}
\noindent
Now, as discussed in \cref{sec:surface-rasterization} -- \enquote{\nameref{sec:surface-rasterization}}, the goal is now to convert these model predictions, which belong to the raster domain $R$, back to the vector polygon domain $V$ by constructing a suitable pseudoinverse mapping $m^{\dagger}: R \rightarrow V$ as illustrated in \cref{fig:pseudoinverse}.
\begin{figure}[H]
  \centering
  \includegraphics{pseudoinverse.tikz}
  \caption{Pseudoinverse mapping, $m^{\dagger}$, mapping from raster domain to vector domain.}%
  \label{fig:pseudoinverse}
\end{figure}
\noindent
We will construct a pseudoinverse mapping which accepts a predicted surface normal raster and semantic segmentation map, and produces predicted three-dimensional polygons, which we will denote as $\predpolygons$,
\begin{equation*}
  \predpolygons
  =
  \pseudoinverse\left(
    \predsegraster,
    \prednormraster
  \right).
\end{equation*}
This pseudoinverse mapping needs to produce three-dimensional polygons that are considered close to the ground truth polygons $\polygons$.
The \enquote{closeness} is encoded in a distance metric $d: V \times V \rightarrow \mathbb{R}$, so our task is to produce a pseudoinverse mapping which minimizes
\begin{equation*}
  d\left(
    \predpolygons,
    \polygons
  \right)
  =
  d\left(
    \pseudoinverse\left(\predsegraster, \prednormraster\right),
    \polygons
  \right).
\end{equation*}
Now, the model predictions $\predsegraster$ and $\prednormraster$ are not perfect, and this must be accounted for in the implementation of the pseudoinverse.
The pseudoinverse must be robust relative to the types of errors commonly produced by the models, producing minimal error as defined by the distance metric $d$.
This is therefore a task that is highly dependent on the behaviour of the trained models themselves.
It is still a good sanity check to test the ground truth rasters, $m(\polygons)$, as well, with other words verify that pseudoinverse mapping also minimizes
\begin{equation*}
  d\left(
    \pseudoinverse\left(m\left(\polygons\right)\right),
    \polygons
  \right).
\end{equation*}
A pseudoinverse mapping which performs badly under this \enquote{round-trip metric} will likely perform badly under the pseudoinverse mapping of model predictions as well.

In \cref{sec:surface-normal-raster-format} we proposed the following conceptual definition of the pseudoinverse mapping
\begin{framed}
  \noindent
  \textbf{Proposed $\mathbf{m^{\dagger}}$} -- Partition semantic segmentation raster $S$ such that all partitions form contiguous pixel areas which share the same value of $N$.
  Reconstruct $\beta_0$ by inference from LiDAR input data.
\end{framed}
\noindent
The implementation of such a pseudoinverse mapping is what will be presented in the following sections.
The conceptual definition was a simplification when presented in the first place, since pixel areas which \enquote{share the same value of $\normraster$} must account for some leeway in the specific definition of \enquote{share}.
The task of determining pixels $(i, j)$ which share the same values for $N_{i, j}$ is considered to be a \textit{clustering problem}, the exact solution of which will be presented in \cref{sec:instance-clustering}.
A divide-and-conquer approach, by the application of \textit{connected region labeling}, will make this problem simpler to solve.
These classified pixels must then be converted to polygons, and will be described in \cref{sec:vectorization}, and possibly simplified as well, as described in the following \cref{sec:simplification}.

Before delving into the details of the implementation, we begin by summarizing the entire post-processing algorithm in conceptual terms:

\begin{leftbar}
  \noindent
  Given a predicted segmentation map $\predsegraster$ and surface normal raster $\prednormraster$:
  \begin{itemize}[leftmargin=*]
    \item Threshold segmentation activations $\predsegraster$ according to some tolerance $\texttt{TOL}$ in order to construct \emph{binary} segmentation raster $\predbinarysegraster$.
    \item Identify connected sub-regions of the binary segmentation raster $\predbinarysegraster$, and sub-divide the following processing independently across these sub-regions.
    \item Apply a conservative clustering algorithm which does \emph{not} require the specification of the number of clusters \textit{a priori}.
    \item Apply a second clustering algorithm which encodes suitable domain knowledge in order to classify the remaining points, producing a labeled partition of the original binary segmentation raster.
    \item Create vector polygons which represent the labeled partitions.
    \item Simplify these vector polygons.
    \item Reconstruct $z$-coordinates of each polygon vertex.
  \end{itemize}
\end{leftbar}
