The clustering performed in the previous section has produced an instance segmentation map, $\predinstraster$.
Lets denote the total number of identified clusters across all roof partitions as $C$ such that $\predinstraster \in \{0, 1, 2, \ldots, C\}^{H \times W \times 1}$.
Here $\predinstraster_{i,j} = 0$ indicates that the pixel location $(i,j)$ does not contain any surface polygon, i.e.\ $\predsegraster_{i,j} = 0$.
This section will describe how to convert this two-dimensional instance segmentation raster map into a set of at least $C$ vectorized three-dimensional polygons $\predpolygons$.

\subsection{Two-dimensional polygonization}

As previously described in \cref{sec:connected-region-labeling}, we can use connected region labeling in order to group together all neighbouring pixels with equal values for $\predinstraster_{i,j}$.
Each connected sub-region can be naively converted to polygons by drawing line segments along pixel borders, resulting in a fine-grained polygon for each instance\footnote{This functionality is implemented in \texttt{GDALPolygonize} in the GDAL library. The \texttt{rasterio} python package wraps this GDAL function in \texttt{rasterio.features.shapes}.}.
Denote the resulting set of two-dimensional polygons as $\predpolygons$, an example of which is shown in \cref{fig:polygonization}.
\begin{figure}[H]
  \centering
  \includegraphics{polygonization.tikz}
  \caption{The result of applying \texttt{GDALPolygonize} on the clustered instance map $\predinstraster$.}%
  \label{fig:polygonization}
\end{figure}
\noindent
As can be seen in \cref{fig:polygonization}, the resulting polygons $\predpolygon \in \predpolygons$ show a high degree of \enquote{rasterization}, having edges which are exclusively oriented along the east-west or north-south axis (\enquote{Manhattan edges}).
Depending on the application of the predicted polygons, it may be preferable to simplify these edges to some degree in order to represent diagonal edges in a better way.
The following section will describe such a simplification.
\newpage

\subsection{Simplification}%
\label{sec:simplification}

A vectorized polygon is, as previously mentioned, simply a collection of line segments which start and end at the same point (a so-called linear ring).
The simplification of vectorized line segments is a well researched area with many proposed algorithms.
The most common approach is to select a subsequence of the original linear ring while minimizing some loss criterion.
Methods for preserving areas \cite{ls-area-preserving}, distances \cite{ls-distance-preserving}, and angles \cite{ls-angle-preserving} have been proposed.
\citeauthor{ls-vw} proposes an area-based progressive algorithm in \cite{ls-vw}\footnote{The Visvalingam-Whyatt polygon simplification method has been implemented and newly released on the Python Package Index (PyPI) under the name \texttt{visvalingamwyatt}.}.
Another common simplification guideline is the \textit{bandwidth criterion}, stating that a simplified line is acceptable as long as it is within an $\varepsilon$-neighbourhood of the original line segment.
The task is then to select the minimum number of vertices which still satisfies this criterion.
The \textit{Ramer-Douglas-Peucker} (RDP) algorithm\cite{ls-rdp} is one such method, a recursive algorithm which is implemented along the following lines:
\begin{pseudofunc}{RDP}{$[p_1, p_2, \ldots, p_n];~\varepsilon$}
  \item Find $p_{\perp} \in [p_1, \ldots, p_n]$ furthest away from the line between $p_1$ and $p_n$.
  \item If the distance from $p_{\perp}$ to the line is less than $\varepsilon$, return $[p_1, p_n]$.
  \item Otherwise, merge result of $\texttt{RDP}([p_1, \ldots, p_{\perp}])$ and $\texttt{RDP}([p_{\perp}, \ldots, p_n])$.
\end{pseudofunc}
An illustration of the RDP algorithm is provided in \cref{fig:rdp-algorithm} on \cpageref{fig:rdp-algorithm}.
We will use a slightly modified version of RDP implemented in the GEOS library which preserves certain topological properties of the simplified polygons\footnote{Documentation for \texttt{TopologyPreservingSimplifier} available at: \url{https://geos.osgeo.org/doxygen/classgeos_1_1simplify_1_1TopologyPreservingSimplifier.html}. A python wrapper is available in the \texttt{simplify()} function implemented in the \texttt{shapely} library when parametrized with \texttt{preserve\_topology=True}: \url{https://shapely.readthedocs.io/en/latest/manual.html\#object.simplify}.}, specifically the number of exterior and interior rings.
The result of applying RDP on the vectorized polygons $\predpolygons$ with $\varepsilon = \SI{0.75}{\meter}$, producing simplified polygons $\simplifiedpolygons$, is shown in \cref{fig:polygon-simplification-tile}.
\begin{figure}[H]
  \centering
  \includegraphics[]{polygon-simplification.tikz}
  \caption{Polygons before and after RDP has been applied with bandwidth tolerance $\varepsilon = \SI{0.75}{\meter}$.}%
  \label{fig:polygon-simplification-tile}
\end{figure}
\begin{figure}[p]
  \centering
  \includegraphics{rdp-algorithm.tikz}
  \appcaption{%
    Illustration of the Ramer-Douglas-Peucker algorithm.
  }{%
    The perpendicular distances between the points $p_{\perp}$ and the lines formed by the endpoints are shown in \textcolor{orange}{orange}.
    Row 1 shows the original line, while rows 2 to 4 show the vertices determined to be included since they are closer than $\varepsilon$.
    Row 5 shows the final resulting line, where the \textcolor{darkgreen}{green} points indicate the vertices to be kept, while the \textcolor{red}{red} points indicate the discarded vertices.
    This specific example has been recreated from Figure~5 in~\cite{original-rdp-figure}.
  }%
  \label{fig:rdp-algorithm}
\end{figure}

\subsection{Three-dimensional fitting}

TODO.
