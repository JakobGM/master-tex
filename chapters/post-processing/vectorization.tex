The clustering performed in the previous section has produced an instance segmentation map, $\predinstraster$.
Lets denote the total number of identified clusters across all roof partitions as $C$ such that $\predinstraster \in \{0, 1, 2, \ldots, C\}^{H \times W \times 1}$.
Here $\predinstraster_{i,j} = 0$ indicates that the pixel location $(i,j)$ does not contain any surface polygon, i.e.\ $\predsegraster_{i,j} = 0$.
This section will describe how to convert this two-dimensional instance segmentation raster map into a set of vectorized three-dimensional polygons $\predpolygons$.

\subsection{Two-dimensional polygonization}

As previously described in \cref{sec:connected-region-labeling}, we can use connected region labeling in order to group together all neighbouring pixels with equal values for $\predinstraster_{i,j}$.
Each connected sub-region can be naively converted to polygons by drawing line segments along pixel borders, resulting in a fine-grained polygon for each instance\footnote{This functionality is implemented in \texttt{GDALPolygonize} in the GDAL library. The \texttt{rasterio} python package wraps this GDAL function in \texttt{rasterio.features.shapes}.}.
Denote the resulting set of two-dimensional polygons as $\predpolygons$, an example of which is shown in \cref{fig:polygonization}.
\begin{figure}[H]
  \centering
  \includegraphics{polygonization.tikz}
  \caption{The result of applying \texttt{GDALPolygonize} on the clustered instance map $\predinstraster$.}%
  \label{fig:polygonization}
\end{figure}
\noindent
As can be seen in \cref{fig:polygonization}, the resulting polygons $\predpolygon \in \predpolygons$ show a high degree of \enquote{rasterization}, having edges which are exclusively oriented along the east-west or north-south axis (\enquote{Manhattan edges}).
Depending on the application of the predicted polygons, it may be preferable to simplify these edges to some degree in order to represent diagonal edges in a better way.
The following section will describe such a simplification.
\newpage

\subsection{Simplification}%
\label{sec:simplification}

\textit{Ramer–Douglas–Peucker algorithm}.

\begin{figure}[H]
  \centering
  \includegraphics[]{polygon-simplification.tikz}
  \caption{Polygons before and after RDP has been applied.}
\end{figure}

\begin{figure}[p]
  \centering
  \includegraphics{rdp-algorithm.tikz}
  \appcaption{%
    Illustration of the Ramer-Douglas-Peucker algorithm.
  }{%
    This specific example has been recreated from Figure~5 in~\cite{original-rdp-figure}.
  }
  \label{}
\end{figure}

\subsection{Three-dimensional fitting}

TODO.
