The clustering described in the previous section results in a rasterized instance segmentation map which we will denote as $\predinstraster$.
Denote the total number of identified clusters across all roof partitions as $C$ such that $\predinstraster \in \{0, 1, 2, \ldots, C\}^{H \times W \times 1}$.
Here $\predinstraster_{i,j} = 0$ indicates that the pixel location $(i,j)$ does not contain any surface polygon, i.e.\ $\predsegraster_{i,j} = 0$.
This section will describe how to convert this two-dimensional instance segmentation raster map into a set of vectorized three-dimensional polygons $\predpolygons$.

\subsection{Two-dimensional polygonization}

Connected region labeling can be used in order to group together all neighbouring pixels with equal values for $\predinstraster_{i,j}$ as previously described in \cref{sec:connected-region-labeling}.
Each connected sub-region can be converted to polygons by simply drawing line segments along the pixel borders, resulting in a fine-grained polygon for each instance\footnote{This functionality is implemented in \texttt{GDALPolygonize} in the GDAL C-library. The \texttt{rasterio} python package wraps this GDAL function in \texttt{rasterio.features.shapes}.}.
Denote the resulting set of two-dimensional polygons as $\predpolygons$, an example of which is shown in \cref{fig:polygonization}.
\begin{figure}[H]
  \centering
  \includegraphics{polygonization.tikz}
  \caption{The result of applying \texttt{GDALPolygonize} on the clustered instance map $\predinstraster$.}%
  \label{fig:polygonization}
\end{figure}
\noindent
As can be seen in \cref{fig:polygonization}, the resulting polygons show a high degree of \enquote{rasterization}, having edges which are exclusively oriented along the east-west or north-south axis (\enquote{Manhattan edges} if you will).
Depending on the application of the predicted polygons, it may be preferable to smooth these edges in order to represent diagonal edges more accurately.
The following section will describe such a simplification.
\newpage

\subsection{Simplification}%
\label{sec:simplification}

A vectorized polygon is, as previously described, simply a collection of line segments which start and end at the same point (a so-called linear ring).
The simplification of vectorized line segments is a well researched area with many proposed algorithms.
The most common approach is to select a subsequence of the original linear ring while minimizing some loss criterion.
Methods for preserving areas~\cite{ls-area-preserving}, distances~\cite{ls-distance-preserving}, and angles~\cite{ls-angle-preserving} have been proposed.
\citeauthor{ls-vw} proposes an area-based progressive algorithm in~\cite{ls-vw}\footnote{The Visvalingam-Whyatt polygon simplification method has been implemented and newly released on the Python Package Index (PyPI) under the name \texttt{visvalingamwyatt}.}.
Another common simplification guideline is the \textit{bandwidth criterion}, stating that a simplified line is acceptable as long as it is within an $\varepsilon$-neighbourhood of the original line segment.
The task is then to select the minimum number of vertices which still satisfies this criterion.
The \textit{Ramer-Douglas-Peucker} (RDP) algorithm\cite{ls-rdp} is one such method, a recursive algorithm which is implemented along the following lines:
\begin{pseudofunc}{RDP}{$[p_1, p_2, \ldots, p_n];~\varepsilon$}
  \item Find $p_{\perp} \in [p_1, \ldots, p_n]$ furthest away from the line between $p_1$ and $p_n$.
  \item If the distance from $p_{\perp}$ to the line is less than $\varepsilon$, return $[p_1, p_n]$.
  \item Otherwise, merge result of $\texttt{RDP}([p_1, \ldots, p_{\perp}])$ and $\texttt{RDP}([p_{\perp}, \ldots, p_n])$.
\end{pseudofunc}
An illustration of the RDP algorithm is provided in \cref{fig:rdp-algorithm} on \cpageref{fig:rdp-algorithm}.
We will use a slightly modified version of RDP implemented in the GEOS library which preserves certain topological properties of the simplified polygons\footnote{Documentation for \texttt{TopologyPreservingSimplifier} available at: \url{https://geos.osgeo.org/doxygen/classgeos_1_1simplify_1_1TopologyPreservingSimplifier.html}.\\A python wrapper is available in the \texttt{simplify()} function implemented in the \texttt{shapely} library when parametrized with \enquote{\texttt{preserve\_topology=True}}: \url{https://shapely.readthedocs.io/en/latest/manual.html\#object.simplify}.}, specifically the number of exterior and interior rings.
The result of applying RDP on the vectorized polygons $\predpolygons$ with $\varepsilon = \SI{0.75}{\meter}$, producing simplified polygons $\simplifiedpolygons$, is shown in \cref{fig:polygon-simplification-tile}.
\begin{figure}[H]
  \centering
  \includegraphics[]{polygon-simplification.tikz}
  \caption{Polygons before and after RDP has been applied with bandwidth tolerance $\varepsilon = \SI{0.75}{\meter}$.}%
  \label{fig:polygon-simplification-tile}
\end{figure}
\begin{figure}[p]
  \centering
  \includegraphics{rdp-algorithm.tikz}
  \appcaption{%
    Illustration of the Ramer-Douglas-Peucker algorithm.
  }{%
    The perpendicular distances between the points $p_{\perp}$ and the lines formed by the endpoints are shown in \textcolor{orange}{orange}.
    Row 1 shows the original line, while rows 2 to 4 show the vertices determined to be included since they are further away than $\varepsilon$.
    Row 5 shows the final resulting line, where the \textcolor{darkgreen}{green} points indicate the vertices to be kept, while the \textcolor{red}{red} points indicate the discarded vertices.
    This specific example has been recreated from Figure~5 in~\cite{original-rdp-figure}.
  }%
  \label{fig:rdp-algorithm}
\end{figure}

\subsection{Three-dimensional reconstruction}

As discussed in \cref{sec:surface-normal-raster-format}, any planar three-dimensional polygon $P$ can be decomposed into, and reconstructed from, two sub-components
\begin{enumerate}[nosep]
  \item the two-dimensional projection $\project{P}$, \dots
  \item and the parametric equation of the plane $\planar{P} = [\beta_0, \beta_x, \beta_y]$.
\end{enumerate}
The first sub-component has now been approximated by the simplified polygons $\simplifiedpolygons$.
We must now try to approximate $\planar{P}$ before we can reconstruct the three-dimensional polygons.
Start by noticing that $\beta_x$ and $\beta_y$ can be expressed in terms of the normal vector of the plane, $\vec{n} = {[n_x, n_y, n_z]}^T$, by construction
\begin{equation*}
  \begin{bmatrix}
    n_x\\
    n_y\\
    n_z
  \end{bmatrix}
  \defeq
  \frac{1}{\sqrt{\beta_x^2 + \beta_y^2 + 1}}
  \begin{bmatrix}
    -\beta_x\\
    -\beta_y\\
    1
  \end{bmatrix}
  \iff
  \begin{bmatrix}
    \beta_x\\
    \beta_y
  \end{bmatrix}
  =
  -\frac{1}{n_z}
  \begin{bmatrix}
    n_x\\
    n_y
  \end{bmatrix}.
\end{equation*}
With other words, if we are able to construct a good estimate for the normal vector of the plane $\normalplanar{P}$, we can also estimate $\beta_x$ and $\beta_y$.
Now, each polygon $P \in \simplifiedpolygons$ has an associated cluster label value $l$ in the instance segmentation map $\predinstraster$.
The corresponding raster index for the given polygon $P$ is then defined as $\mathcal{I}_P \defeq \{(i, j) \mid \predinstraster_{i,j} = l \}$, and the predicted normal vectors associated to the polygon $P$ can therefore be retrieved from the predicted surface normal vector raster as $\mathcal{N}_{P} \defeq \{\predsegnormraster_{i,j} \mid (i,j) \in \mathcal{I}_P \}$.
We can now construct a predictor for $\normalplanar{P}$ from the associated normal vector set $\mathcal{N}_P$, for example the \textit{normalized vector average}, defined as
\begin{equation*}
  \overline{\vec{n}}
  =
  \begin{bmatrix}
    \overline{n}_x\\
    \overline{n}_y\\
    \overline{n}_z
  \end{bmatrix}
  \defeq
  \frac{
    \sum_{\predsegnormraster_{i,j} \in \mathcal{N}_P} \predsegnormraster_{i,j}
  }{
    \norm{\sum_{\predsegnormraster_{i,j} \in \mathcal{N}_P} \predsegnormraster_{i,j}}_2
  }.
\end{equation*}
It is also possible to remove the DBSCAN-identified noise from the normal vector set in order to construct an even more robust estimator, that is, if the polygon $P$ belongs to roof partition $r$, we use $\mathcal{N}_{P}^* \defeq \{\predsegnormraster_{i,j} \mid (i,j) \in (\mathcal{I}_P \setminus \mathcal{I}_r^*) \}$ instead.
We can now construct estimators for $\beta_x$, $\hat{\beta}_x \defeq -\overline{n}_x / \overline{n}_z$, and $\beta_y$, $\hat{\beta}_y \defeq -\overline{n}_y / \overline{n}_z$.
The remaining task is now to construct an estimator for the final planar parameter $\beta_0$.
Since the surface normal raster is entirely independent of the elevation of the plane, we must use the LiDAR input data in order to construct an estimator $\hat{\beta}_0$ for $\beta_0$.
The idea is to formulate the search for $\hat{\beta}_0$ such that the \textit{LiDAR residuals} (see \cref{sec:non-planar-fix}) are minimized.
Given a distance metric $d$, we can construct the predictor as
\begin{align*}
  \hat{\beta}_0
  &=
  \argmin_{\beta_0 \in \mathbb{R}}
  \sum_{(i,j) \in \mathcal{I}_P}
  d\left(
    \lidarraster_{i,j},~
    \beta_0 + \hat{\beta}_x x + \hat{\beta}_y y
  \right),
\end{align*}
where ${[x, y]}^T \defeq \pixtogeo{i}{j}$ and $H_{i,j}$ is the LiDAR measurement at pixel location $(i,j)$.
If we specify the square distance metric $d(x, y) = (x - y)^2$, then we have an ordinary least squares regression with just one regression parameter.
Due to the nature of LiDAR measurement errors, the absolute distance $d(x, y) = |x - y|$ has been shown to produce more robust estimators for $\beta_0$.

We can now reconstruct the predicted set of simplified, three-dimensional polygons:
\begin{equation*}
  P_{\mathrm{3D}}
  =
  \left[
    (x, y, \hat{\beta}_0 + \hat{\beta}_x x + \hat{\beta}_y y)
    \mid
    \text{ for vertex } (x, y) \text{ representing } P
  \right],
  \text{ for } P \in \simplifiedpolygons.
\end{equation*}
