The predicted segmentation map $\predsegraster$ produced by $\segmodel$ is a $H \times W \times 1$ raster array where the values $0 \leq \predsegraster_{i,j} \leq 1$ can be interpreted as the model's confidence in there being a roof surface at pixel location $(i, j)$.
In order to make an actual binary prediction for each pixel, we \textit{threshold} the segmentation map, creating a binary segmentation map $\predbinarysegraster$ defined by
\begin{equation*}
  \predbinarysegraster \defeq \predsegraster > \texttt{TOL}
  \iff
  \predbinarysegraster_{i, j}
  =
  \begin{cases}
    1, &\text{if } \predsegraster_{i, j} > \texttt{TOL}. \\
    0, &\text{otherwise.}
  \end{cases}
\end{equation*}
We will use the most commonly used threshold value, namely $\texttt{TOL} = 0.5$.
The process of thresholding has been illustrated in \cref{fig:activation-thresholding}.
\begin{figure}[H]
  \centering
  \resizebox{0.68\textwidth}{!}{\includegraphics{activation-thresholding.tikz}}
  \caption{Thresholding a probabilistic segmentation prediction in order to create a \textit{binary} segmentation map.}%
  \label{fig:activation-thresholding}
\end{figure}
\noindent
This binary segmentation map can now be used in order to segment the predicted surface normal raster as well, producing what we will refer to as a \textit{segmented surface normals}.
The segmented surface normal raster, $\predsegnormraster$, is constructed by taking the element-wise product of the predicted surface normal raster and the thresholded binary segmentation map, which we will denote as $\predbinarysegraster \odot \prednormraster$ and formally define as
\begin{equation*}
  \predsegnormraster \defeq \predbinarysegraster \odot \prednormraster
  \iff
  \predsegnormraster_{i, j, d}
  =
    \predbinarysegraster_{i, j}
    \cdot
    \prednormraster_{i, j, d},
  \text{ for } d \in \{x, y, z\}.
\end{equation*}
The segmentation of the predicted surface normal raster is illustrated in \cref{fig:normal-segmentation}.
\begin{figure}[H]
  \centering
  \includegraphics{normal-segmentation.tikz}
  \caption{Segmentation of a predicted surface normal raster using a binary segmentation map.}%
  \label{fig:normal-segmentation}
\end{figure}
\noindent
These segmented normals $\predsegnormraster$ are intended to be used in order to partition the semantic segmentation map $\predbinarysegraster$ into an instance segmentation map, where each instance is considered to be a projected roof surface polygon.
A simple first-pass partitioning of the segmentation map into regions which are entirely connected can be performed, with other words the partitioning into entire roofs rather than single roof surfaces.
The identification of connected sub-regions can be performed by a so-called \textit{connected region labeling} algorithm as described in \cite{sklabel1,sklabel2}, and has been implemented by the \texttt{skimage} python package in \texttt{skimage.measure.label}.
\begin{figure}[H]
  \centering
  \includegraphics{mask-iteration.tikz}
  \caption{The result of \texttt{skimage.measure.label} applied on a binary segmentation map.}%
  \label{fig:mask-iteration}
\end{figure}
\noindent
A dynamic programming programming approach can be subsequently applied in order to subdivide these \textit{roof partitions} into \textit{surface partitions}.
The example presented in \cref{fig:mask-iteration} shows that there exists four separate connected regions ($A$, $B$, $C$, and $D$) in the predicted semantic segmentation raster.
Provided that the predicted segmentation map is sufficiently correct, then we can conclude that any polygon belonging to any one of these regions will not belong to any other region.
We can therefore formulate the partitioning of each region as entirely independent tasks, the solution of which will be presented in the next section.
