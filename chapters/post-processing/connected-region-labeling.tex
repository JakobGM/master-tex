The predicted segmentation map $\predsegraster$ produced by $\segmodel$ is a $H \times W \times 1$ raster array with values in the continuous domain $[0, 1]$, values $\predsegraster_{i,j}$ which can be interpreted as the model's confidence in there being a roof surface at pixel location $(i, j)$.
In order to make an actual binary prediction, we \textit{threshold} the segmentation map, creating a binary segmentation map $\predbinarysegraster$ defined by
\begin{equation*}
  \predbinarysegraster \defeq \predsegraster > \texttt{TOL}
  \iff
  \predbinarysegraster_{i, j}
  =
  \begin{cases}
    1, &\text{if } \predsegraster_{i, j} > \texttt{TOL}. \\
    0, &\text{otherwise.}
  \end{cases}
\end{equation*}
We will use the most commonly used threshold value, naturally $\texttt{TOL} = 0.5$.
The process of thresholding has been illustrated in \cref{fig:activation-thresholding}.
\begin{figure}[H]
  \centering
  \includegraphics{activation-thresholding.tikz}
  \caption{Thresholding model segmentation prediction in order to create a \textit{binary} segmentation map.}%
  \label{fig:activation-thresholding}
\end{figure}
\noindent
This binary segmentation map can now be used in order to segment the predicted surface normal raster as well, producing what we will refer to as a \textit{segmented surface normals}.
The segmented surface normal raster can be easily produced by taking the element-wise product of the predicted surface normal raster and the thresholded binary segmentation map, which we will denote as $\predbinarysegraster \odot \prednormraster$, and is formally defined as
\begin{equation*}
  \predsegnormraster \defeq \predbinarysegraster \odot \prednormraster
  \iff
  \predsegnormraster_{i, j}
  =
    \predbinarysegraster_{i, j}
    \cdot
    \left[\prednormraster_{i,j,x}, \prednormraster_{i,j,y}, \prednormraster_{i,j,z}\right].
\end{equation*}
The effect of segmenting the predicted surface normal raster is illustrated in \cref{fig:normal-segmentation}.
\begin{figure}[H]
  \centering
  \includegraphics{normal-segmentation.tikz}
  \caption{Demonstration of segmenting a predicted surface normal raster using a thresholded segmentation map.}%
  \label{fig:normal-segmentation}
\end{figure}
\noindent
We have now found an index set, $\mathcal{I}$, for which there is predicted to contain a polygon.
\begin{equation*}
  \mathcal{I}
  =
  \left\{
  (i, j)
  ~\big{|}~
  \left|\predpolygons\left(\pixtogeo{i}{j}\right)\right| > 0
  \right\}
\end{equation*}
We now need to assign a single label to each such coordinate, indicating which polygon $P \in \predpolygons$ the given coordinate belongs to.
We apply a divide-and-conquer approach by partitioning the thresholded segmentation map into those regions which are connected.
This can be performed by a so-called \textit{connected region labeling} algorithm, and has been illustrated in \cref{fig:mask-iteration}.
\begin{figure}[H]
  \centering
  \includegraphics{mask-iteration.tikz}
  \caption{The result of the  \textit{connected region labeling} algorithm applied on the thresholded segmentation map.}%
  \label{fig:mask-iteration}
\end{figure}
\noindent
The example presented in \cref{fig:mask-iteration} shows that there exists four, separate connected regions ($A$, $B$, $C$ and $D$).
If we assume that the segmentation map is correct, then we can conclude that any polygon belonging to one of these regions will not belong to any other region.
We can therefore solve the partition of each region as entirely independent tasks.
The partitioning of each connected sub-region is now formulated as a normal vector clustering problem.
