Geographic data is in wide use by both the public and private sector, and is a huge subject in and of itself.
The storage, processing, and inspection of such data is handled by \textit{Geographic Information Systems} (GIS).
In this section we will explain a few core GIS concepts relevant for the problem at hand, concepts which will inform decisions for how to prepare the data for machine learning purposes.
\Cref{sec:coordinate-systems} will give a brief introduction to the coordinate systems used to represent geographic data.
GIS data can be largely bisected into two categories, vector data and raster data, and both types will be described in \cref{sec:data-types}.
\Cref{sec:data-sets} will present the datasets used for training our models.
The remaining subsections will describe the pre-processing pipeline which has been developed for our specific purposes, preprocessing in the form of cadastral tiling (\cref{sec:tiling-algorithm}), segmentation masking (\cref{sec:masking-algorithm}), and surface rasterization (\cref{sec:surface-rasterization}).
A figurative overview of the preprocessing pipeline is provided by \cref{fig:preprocessing-overview} in \cref{app:preprocessing-overview}.
