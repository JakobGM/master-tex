Raster data consists of a set of scalar measurements imposed onto a grid.
A color image, $I$, of width $\mathrm{W}$ and height $\mathrm{H}$, will contain three color channels; red, green, and blue (RGB), and can be represented by a three-dimensional array of size $\mathrm{H} \times \mathrm{W} \times \mathrm{3}$.
Each color channel for a given pixel is represented by an unsigned 8-bit integer, i.e.
%
\begin{equation*}
  I_{i, j, c} \in \{0, 1, \ldots, 255\},
  \hspace{2.5em}
  i = 1, \ldots, \mathrm{H} ,
  ~~
  j = 1, \ldots, \mathrm{W},
  ~~
  c = \mathrm{\textcolor{red}{1}, \textcolor{green}{2}, \textcolor{blue}{3}}.
\end{equation*}
%
A LiDAR elevation map, which we will denote as $Z$, is likewise encoded as a single-channel grayscale image of size $\mathrm{W} \times \mathrm{H}$.
Each pixel is represented by a signed 32-bit floating point value which gives the following approximate value domain
%
\begin{equation*}
  Z_{i, j} \in \mathbb{R},
  \hspace{2.5em}
  i = 0, \ldots, \mathrm{H} - 1,
  ~~
  j = 0, \ldots, \mathrm{W} - 1.
\end{equation*}
%
These two raster types must be handled differently during data-standardization and -normalization due to their different value domains, which we will come back to in \cref{sec:raster-normalization}.

For GIS rasters specifically we must additionally provide the spatial extent of the given raster defined by:
\begin{itemize}[noitemsep]
  \item A coordinate system, for example UTM \texttt{32V}.
  \item The coordinate of the center of the upper left pixel, $I_{1, 1}$; the \textit{origin} $\vec{r}_0 = {[x_0, y_0]}^T$.
  \item The pixel step size, $\vec{\Delta} = {[\Delta_x, \Delta_y]}^T$, for example ${[\SI{0.25}{\meter}, \SI{-0.25}{\meter}]}^T$.
\end{itemize}

The pixel $I_{i, j, c}$ therefore represents a rectangle of width $\Delta_x$ and height $\Delta_y$ centered at the spatial coordinate $\vec{r}_0 + \vec{\Delta} \cdot [i, j]$ interpreted in the given coordinate system.

Missing data in remote sensing rasters is specified by filling in a predefined \texttt{nodata} placeholder value.
For RGB data this is often set to $0$, resulting in a black pixel.
LiDAR rasters often use $\texttt{nodata} = -2^{127} \times (2 - 2^{-23}) \approx -3.4028234664 \times 10^{38}$, the most negative normal number representable by a single-precision floating point number.
Such \texttt{nodata} values may arise from measurement errors or by pixels situated outside the given coverage area of the dataset, and must be special-cased during data normalization, which we will come back to in \cref{sec:raster-normalization}.

When we will train models on the combination of LiDAR and aerial photography data, these two types of rasters must be merged in order to attain a consistent three-dimensional array of size $H \times W \times 4$.
These rasters can not be simply superposed when their pixel sizes $\vec{\Delta}$ and/or origins $\vec{r}_0$ differ.
In such cases we will apply bilinear interpolation on the raster of greatest resolution and subsequently downsample it in order to align all pixels.
See \cref{app:raster-merging} for how this is performed in practice.
