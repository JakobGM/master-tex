The \enquote{Matrikkelen - Eiendomskart Teig}\footnote{Product specification for \enquote{Matrikkelen - Eiendomskart Teig} can be found here:\\\url{https://kartkatalog.geonorge.no/metadata/74340c24-1c8a-4454-b813-bfe498e80f16}.} data set contains all cadastral plots in Trondheim, the use of which will be explained in \cref{sec:tiling-algorithm}.
The \enquote{FKB-bygning}\footnote{Product specification for \enquote{FKB-bygning} can be found here:\\\url{https://kartkatalog.geonorge.no/metadata/8b4304ea-4fb0-479c-a24d-fa225e2c6e97}.} data set contains all registered building outlines in Trondheim.
The building outlines will be used to construct binary classification masks as outlined in \cref{sec:masking-algorithm}.
Both data sets are illustrated in \cref{fig:vector-data-example}.

\begin{figure}[htb]
  \includegraphics[trim={5cm 0 5cm 0},clip, width=0.49\linewidth]{data/teig-example}
  \includegraphics[trim={5cm 0 5cm 0},clip, width=0.49\linewidth]{data/building-example}
  \caption{%
    Illustration of vector data sets.
    Cadastral plots are shown on the left while building outlines are shown on the right.
    \copyright{Kartverket}.
  }%
  \label{fig:vector-data-example}
\end{figure}
