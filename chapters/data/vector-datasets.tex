The \enquote{Matrikkelen - Eiendomskart Teig}\footnote{Product specification for \enquote{Matrikkelen - Eiendomskart Teig} can be found here:\\\url{https://kartkatalog.geonorge.no/metadata/74340c24-1c8a-4454-b813-bfe498e80f16}.} dataset contains all cadastral plots in Trondheim, the use of which will be explained in \cref{sec:tiling-algorithm}.
The \enquote{FKB-bygning}\footnote{Product specification for \enquote{FKB-bygning} can be found here:\\\url{https://kartkatalog.geonorge.no/metadata/8b4304ea-4fb0-479c-a24d-fa225e2c6e97}.} dataset contains all registered building outlines in Trondheim.
The building outlines will be used to construct binary classification masks as outlined in \cref{sec:masking-algorithm}.
Additionally, the \enquote{FKB-bygning} dataset includes a complete collection of descriptive building lines, such as ridge lines, verge lines, and so on.
This dataset is of no direct use to us, but Norkart has constructed an algorithm for merging these lines in order to construct three-dimensional roof surface polygons.
This polygonized dataset has been kindly provided to us by Norkart, and will be used as outlined in \cref{sec:surface-rasterization}.
These four datasets are illustrated in \cref{fig:vector-data-example}.

\begin{figure}[H]
  \includegraphics[trim={5cm 0 5cm 0},clip, width=0.4\linewidth]{data/teig-example}
  \includegraphics[trim={5cm 0 5cm 0},clip, width=0.4\linewidth]{data/building-example}
  \vspace{5em}
  \includegraphics[width=0.4\linewidth]{ortofoto_linjesegmenter.png}
  \includegraphics[width=0.4\linewidth]{ortofoto_polygons.png}
  \appcaption{%
    Illustration of vector datasets.
  }{%
    Cadastral plots are shown on the top left while building outlines are shown on the top right.
    Descriptive building lines are shown on the bottom left, such as ridge lines shown in \textcolor{red}{red}, while the polygonized roof surfaces produced by Norkart are shown on the bottom right.
    \copyright{Kartverket}.
  }%
  \label{fig:vector-data-example}
\end{figure}
