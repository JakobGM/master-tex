We will use the \enquote{Ortofoto Trondheim 2017}\footnote{Product specification for \enquote{Ortofoto Trondheim 2017} can be found here:\\ \url{https://kartkatalog.geonorge.no/metadata/cd105955-6507-416f-86d2-6d95c1b74278}.} aerial photography dataset from 2017 which requires \SI{161}{\giga\byte} of storage space.
The real image resolution is \SIrange{0.04}{0.15}{\meter}, but is provided with an upsampled resolution of \SI{0.1}{\meter} for consistency.
The reported accuracy is \SI{\pm 0.35}{\meter}~\cite{trondheim_ortophoto_2017}, although the exact type of this accuracy is not specified.
An exemplified region is visualized in \cref{fig:rgb-example}.

\begin{figure}[H]
  \centering
  \includegraphics[width=0.75\linewidth]{data/rgb-example}
  \appcaption{%
    Visualization of the \enquote{Ortofoto Trondheim 2017} aerial photography dataset.
  }{%
    \copyright{Kartverket}.
  }%
  \label{fig:rgb-example}
\end{figure}

An \textit{orthophoto} is an image where the geographic scale is uniform over the entire image.
Proper orthophotos are expensive to manufacture and are therefore seldomly available for most geographic regions~\cite{ortofoto_in_norway_2003}, including Trondheim.
Aerial photography which has not been properly \enquote{ortho-rectified} may impede location-based inference as there exists no exact one-to-one mapping between image pixels and geographic coordinates.
This problem is best understood by an example, as shown in \cref{fig:non-orthophoto-example}.

\begin{figure}[H]
  \centering
  \begin{tikzpicture}
    \node[anchor=south west,inner sep=0] (image) at (0,0) {\includegraphics[width=0.75\linewidth]{data/non-orthophoto-example}};
    \begin{scope}[x={(image.south east)},y={(image.north west)}]
      \draw[orange, ultra thick, fill=orange!50, fill opacity=0.25]
        (0.34, 0.3) --
        (0.615, 0.55) --
        (0.61, 0.69) --
        (0.33, 0.445) --
        cycle;
    \end{scope}
  \end{tikzpicture}
  \appcaption{%
    Example of nonproper orthophoto.
  }{%
    The building centered in the image is 14 stories tall.
    The orange area annotates a clearly visible building wall.
    \copyright{Kartverket}.
  }%
  \label{fig:non-orthophoto-example}
\end{figure}

As can be seen in \cref{fig:non-orthophoto-example}, the \enquote{Ortofoto Trondheim 2017} dataset clearly shows one side of a building due to the perspective of the plane capturing the image.
An ideal orthophoto would capture all vertical building walls as single, straight lines, no matter the perspective.
The effect of this \enquote{parallax error} on segmentation predictions will be investigated in \cref{sec:experiments}\todo{Should we have a specific RGB experiment and reference to it here?}.

The LiDAR dataset used is \enquote{Høydedata Trondheim 5pkt 2017}\footnote{Product specification for \enquote{Høydedata Trondheim 5pkt 2017} can be found here:\\ \url{https://kartkatalog.geonorge.no/metadata/bec4616f-9a62-4ecc-95b0-c0a4c29401dc}.} from \date{2017-10-10} and requires \SI{25}{\giga\byte} of storage space.
The resolution is \SI{0.2}{\meter} and has a reported standard deviation of \SI{0.02}{\meter}~\cite{trondheim_lidar_2017}.
LiDAR visualized as a grayscale image over the same region as in \cref{fig:rgb-example} is presented in \cref{fig:lidar-example}.

\begin{figure}[H]
  \centering
  \includegraphics[width=0.7125\linewidth]{data/lidar-example}
  \caption[Visualization of LiDAR data from Trondheim.]{%
    Visualization of the \enquote{Høydedata Trondheim 5pkt 2017} LiDAR dataset.
    \copyright{Kartverket}.
  }%
  \label{fig:lidar-example}
\end{figure}

The LiDAR dataset is partially incomplete due to measurement errors, and certain pixels are therefore filled in with \texttt{nodata} placeholder values as explained in \cref{sec:data-types}.
\cref{tab:lidar-point-density} shows the frequency of such \texttt{nodata} values in the dataset.
The table has been produced by moving a rolling window of size $\SI{10}{\meter} \times \SI{10}{\meter}$ over the entire coverage area and calculating the proportion of pixels with valid values within each non-overlapping window.

\begin{table}[htb]
  \centering
  \begin{tabular}{cc}
    \toprule
    {Point density (\si{\per\meter\squared})} & {Proportion (\%)} \\
    \midrule
    $> 100\%$ & 97.7 \\
    \SIrange{85}{100}{\percent} & 1.2 \\
    \SIrange{60}{85}{\percent} & 1.1 \\
    \bottomrule
  \end{tabular}
  \caption[Control of point cloud LiDAR density.]{%
    Control of point cloud density for the Trondheim 2017 LiDAR dataset.
    The densities are calculated within rolling windows of size $\SI{10}{\meter} \times \SI{10}{\meter}$~\cite{trondheim_lidar_2017}.
  }%
  \label{tab:lidar-point-density}
\end{table}

% \SI{70.77}{\percent} of all pixels are valid, probably due to lower resolution than the actual resolution.  Elevation data is in domain $(-9.390, 569.050)$.
