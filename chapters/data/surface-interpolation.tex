Let $r_{p,i}$ denote a linear ring belonging to a polygon denoted as $P_p$.
The symbol $p$ denotes the \textit{polygon index}, while $i$ denotes the \textit{ring index}.
A ring index of 1 indicates an \textit{exterior ring}, while a ring index of greater than one represents an interior ring.
The polygon $P_p$ can therefore be represented as a sequence of $|P_p|$ linear rings,
%
\begin{equation*}
P_p = \{ r_{p,1}, \dots, r_{p, |P_p|}\}
\end{equation*}
%
Linear rings are represented as ordered sequences of $(x, y, z)$ coordinate tuples, assuming that the associated polygon is three-dimensional.
The linear ring can therefore be denoted as
%
\begin{equation*}
  r_{p,i}
  =
  \{
    (x_{p,i,1}, y_{p,i,1}, z_{p,i,1}),
    \dots,
    (x_{p,i,|r_{p, i}|}, y_{p,i,|r_{p, i}|}, z_{p,i,|r_{p, i}|}),
    (x_{p,i,1}, y_{p,i,1}, z_{p,i,1})
  \},
\end{equation*}
%
where the first and last coordinate tuple are identical in order to close the ring.

\begin{itemize}
  \item An indexed polygon collection $\mathcal{P}$ containing $|\mathcal{P}|$ polygons:
    \[\mathcal{P} = \{P_1, P_2, \dots, P_{|\mathcal{P}|}\}\]
  \item A given polygon, $P_p \in \mathcal{P}$, of index $p$ is represented by a single exterior ring, denoted as $r_{p,0}$, and $|P_p| - 1$ interior hulls, $\{r_{p,1}, \dots, r_{p,|P_p| - 1}\}$.
    \[P_p = \{ r_{p,0}, \dots, r_{p, |P_p|}\}\]
  \item A given linear ring, $r_{p,i}$, is represented by a sequence $|r_{p,i}| + 1$ $xyz$ coordinate tuples,
\end{itemize}

\begin{equation*}
  X_p
  =
  \begin{bmatrix}
    1 & x_{p,0,0} & y_{p,0,0} \\
    1 & x_{p,0,1} & y_{p,0,1} \\
    \vdots & \vdots & \vdots \\
    1 & x_{p,0,|r_{p0}|} & y_{p,0,|r_{p0}|} \\
    1 & x_{p,1,0} & y_{p,1,0} \\
    1 & x_{p,1,1} & y_{p,1,1} \\
    \vdots & \vdots & \vdots \\
    1 & x_{p,|P_p|,|r_{p}|} & y_{p,1,1} \\
  \end{bmatrix}
\end{equation*}


\begin{figure}
  \centering
  \includegraphics[width=\linewidth]{interpolation-concepts}
  \caption{Three-dimensional surface raster values}{%
    Illustration of interpolated altitude values, $Z_{ij}$, and respective surface normal vectors, $N_{ij}$.
    Slice for $i = 1$ and $1 \leq j \leq 4$.
  }
\end{figure}

\begin{algorithm}{Surface interpolation}{alg:surface-interpolation}{Tile bounding box $B(\vec{c}, w, h)$,\\Raster dimensions $H \times W$,\\R-Tree indexed polygon collection.}
\item Construct arrays with temporary placeholder values:
  \begin{itemize}[label=--,leftmargin=0cm]
    \item Normal vector array $N$ of size $H \times W \times 3$ filled with $0$.
    \item Interpolated elevation array $Z$ of size $H \times W \times 1$ filled with $-\infty$.
  \end{itemize}
\item Construct polygon collection $\mathcal{P}$ for given tile extents using R-Tree index.
  Convert polygons to pixel coordinate system.
\item For polygon $P_p = [r_{p0}, r_{p1}, \dots, r_{pn_p}]$ with a single exterior ring, $r_{p0}$, and $n_p$ interior rings, $[r_{p1}, \ldots, r_{pn_p}]$, in polygon collection $\mathcal{P}$.
  \begin{enumerate}[leftmargin=0.5em,label=\textbf{\alph*}]
      % \item The linear ring $r_{pj} = [(x_{pj0}, y_{pj0}, z_{ij0}), \dots, (x_{pjn_{ij}}, y_{ijn_{pj}}, z_{ijn_{pj}})]$.
    \item Construct $M \times 3$ design matrix $X$ populated with pixel coordinates of all \textit{unique} xy-vertices $(1, x_{pij}, y_{pij})$.
        Secondly, construct $M \times 1$ response vector $\vec{y}$ populated with the respective $z_{pij}$ coordinates.
    \item Solve linear regression problem $\vec{\beta}_p = {\left(\beta_0, \beta_x, \beta_y\right)}^T = {\left(X^T X\right)}^{-1} X^T \vec{y}$
    \item Construct polygon surface normal vector $\vec{n}_p$,
      \begin{equation*}
        \vec{n}_p \assign {\left(\beta_x^2 + \beta_y^2 + 1\right)}^{-1/2} \cdot {\left(-\beta_x, -\beta_y, 1\right)}^T,
      \end{equation*}
      where $||\vec{n}_p||_2 = 1$ by construction.
    \item For each pixel coordinate $(i, j)$ contained by the polygon $P_p$ projected onto the 2-dimensional pixel index plane:
      \begin{itemize}[leftmargin=0.5em]
        \item $h \assign \beta_0 + \beta_x j + \beta_y i$. If $h < Z_ij$, continue loop, else\dots
        \item $Z_{ij} \assign h$ and $\left(N_{ijx}, N_{ijy}, N_{ijz}\right) \assign \vec{n}_p$.
      \end{itemize}
    \end{enumerate}
\end{algorithm}
