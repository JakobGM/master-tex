\begin{wrapfigure}[10]{r}{0.25\textwidth}
  \vspace{-2em}
  \begin{center}
    \includegraphics[width=\linewidth]{3d-roof.tikz}
  \end{center}
  \caption{%
    \\
    Three-dimensional\\polygonal gable roof.
  }%
  \label{fig:gable-roof}
\end{wrapfigure}

The roof of a given building can be decomposed into a collection of entirely flat polygons.
This is an accurate data decomposition in most cases, the exception being conic shapes and other surfaces with continuous curvature which cannot be perfectly represented with a finite set of flat polygons.
A \emph{gable roof}\footnote{\textit{Gable roof} -- A roof which consists of two flat roof surfaces which slope in opposite directions.
The roof surfaces are connected along the highest, horizontal edge (ridge).}, for instance, can be considered as the set of two flat polygons, which when combined together accurately represent the roof in its entirety.
Our intent is to construct a machine learning pipeline which is able to identify such three-dimensional roof surfaces from remote sensing data.

Although a set of vectorized, flat polygons is often the most suitable data representation for geometric roof data in the GIS world, it is \emph{not} considered an ideal representation for traditional machine vision data pipelines.
Polygons can consist of an arbitrary number of linear rings, each with an arbitrary number of vertices.
The number of rings and vertices depends on the complexity of the given polygon's shape.
Deep learning model architectures, on the other hand, are often restricted to training on and predicting observations of \emph{consistent} dimensionality.
This is why machine vision architectures most often accept and/or emit spatial information in the form of \emph{rasters} rather than vectors, that is, numeric arrays of consistent size.
In order to reconcile these conflicting requirements we now pose the following question:
%
\begin{quotation}
  \enquote{How can an arbitrarily sized set of three-dimensional polygons, all of arbitrary complexity, shape, and orientation, be accurately represented in the form of a raster?}.
\end{quotation}
%
The ideal raster format would allow us to train on and predict roof surfaces in vectorized polygon form, all the while applying the tried and tested techniques from the machine vision literature which mainly concerns itself with rasters.
We will refer to such a raster as a \textit{surface raster} in order to distinguish it from other remote sensing raster types such as aerial photography and LiDAR data.
The careful formulation and construction of this surface raster format is considered one of the most important problems to be solved in order to construct an efficient machine learning pipeline for predicting roof surfaces.
\begin{figure}
  \begin{center}
    \includegraphics{invertible-rasterization.tikz}
  \end{center}
  \vspace{-1.5em}
  \caption{Invertible rasterization}
  \label{fig:invertible-rasterization}
  \vspace{-1.5em}
\end{figure}

\subsection{Desirable surface raster properties}
The ideal surface raster format should be both \emph{representative} and \emph{targetable}, properties which we now shall formally define.
Start by denoting the domain which consists of polygon \textit{sets} of arbitrary cardinality as $V$.
A polygon in vector format will be denoted as $P$, while a set of vector polygons will be denoted as $\mathcal{P} = \{P_1, P_2, \dots\}$.
A vector polygon set $\mathcal{P}$ is therefore a member of the superset domain $V$, denoted as $\mathcal{P} \in V$.
Assume that we want to construct a surface raster of height $H$ and width $W$, and that this raster will consists of $C_R$ raster channels.
If we denote this surface raster domain as $R$ and assume that raster values will take values from the real number line then we have $R = \mathbb{R}^{H \times W \times C_R}$.
Finally, assume that the remote sensing raster data, $X$, has all the same dimensional properties as the surface raster data with the possible exception of the number of raster channels, which we will denote as $C_X$, i.e. $X \in \mathbb{R}^{H \times W \times C_X}$.
\begin{description}[style=nextline]
  \item[Representative]
    \textit{Converting surface polygons to the surface raster format and then back again incurs negligible loss of information.}
    \\
    Define a suitable distance metric $d: V \times V \rightarrow{R}$ which incorporates some notion of the difference in spatial location and orientation between two sets consisting three-dimensional polygons.
    A \emph{perfectly} \textit{representative} raster format would allow us to define a mapping from the vector domain to the surface raster domain, $m: V \rightarrow R$, for which there exists a functional inverse:
    \begin{align*}
      m^{-1}\left(m(\mathcal{P})\right)
      &\equiv
      \mathcal{P},
      &\forall \mathcal{P} \in V.
      \\
      \implies d\left(
        m^{-1}(m(\mathcal{P})),~\mathcal{P}
      \right)
      &\equiv
      0,
      &\forall \mathcal{P} \in V.
    \end{align*}
    The vector domain $V$ is infinite-dimensional, while the raster domain $R$ is by necessity of finite and consistent size.
    It can therefore be concluded that no such invertible mapping exists.
    For this reason we introduce the concept of a \textit{pseudo-inverse}, $m^{\dagger}$, a function which minimizes $d(m^{\dagger}(m(\mathcal{P})), \mathcal{P})$ for all $\mathcal{P} \in V$.
    A raster domain (and associated pseudo-invertible mapping) which produces negligible distance metrics is considered to be \textit{representative}.
  \item[Targetable] 
    \textit{A raster format which is feasible as a modelling target for a machine learning architecture.}
    \\
    We intend to construct a predictor, $\hat{f}$, which accepts remote sensing raster data as input and produces the aforementioned surface raster data representation as output, i.e.  $\hat{f}: V \rightarrow R$, or equivalently
    $
      \hat{f}:
        \mathbb{R}^{H \times W \times C_X}
        \rightarrow
        \mathbb{R}^{H \times W \times C_R}
    $.
    Now assume $\hat{f}$ to be parametrized according to the parameter vector $\vec{\theta}$, and denote the parametrized prediction as $\hat{Y} \defeq \hat{f}(X; \vec{\theta})$.
    The intention is that surface raster prediction $\hat{Y}$ generated from remote sensing data should be as similar to the polygon-constructed raster $Y \defeq m(\mathcal{P})$ as possible.
    Similar to the distance metric $d$ defined previously, we now define a suitable differentiable \textit{loss} function $\mathcal{L}: R \times R \rightarrow \mathbb{R}$ which incorporates some notion of difference between two surface rasters.
    The predictor $\hat{f}$ can therefore be parametrized such that this loss function is minimized when evaluated on the predicted surface raster in conjunction with the ground truth surface raster $m(P)$:
    \begin{align*}
      \vec{\theta}_{\mathrm{opt}}
      &\defeq
      \argmin_{\vec{\theta}}
      \sum_{(\hat{Y}, Y)}
      \mathcal{L}\left(
        \hat{Y};
        Y
      \right)
      \\
      &=
      \argmin_{\vec{\theta}}
      \sum_{(X, \mathcal{P})}
      \mathcal{L}\left(
        \hat{f}(X; \vec{\theta});
        m(\mathcal{P})
      \right)
    \end{align*}
    A raster mapping for which a suitable loss function can be constructed and minimized is considered to be \textit{targetable}.
\end{description}
The considerations of representativeness and targetability are in many ways diametrically opposed when constructing a suitable raster mapping.
As an instructive example consider the choice of $C_R$, the number of raster channels used by the surface raster.
Since the mapping $m$ maps from a infinite-dimensional space to a finite-dimensional one, it can be considered a compression method of sorts.
The smaller the value of $C_R$, the greater the compression, and subsequently its associated compression loss.
Thus the greater number of raster channels, the more representative the raster format can become.
On the other hand, when $C_R$ grows large it is natural to assume that the increasing degree of freedom in the data format allows for many equivalently accurate representations of the same polygon collection.
The raster format may also become more sparse as a result.
This ambiguity and sparseness will result in a difficulties when formulating a proper, convex loss function with single, global minima.
The targetability of the raster format may therefore suffer from large values of $C_R$.

Although it is the surface raster loss function we minimize in practice, it is actually not what we \emph{really} are interested in.
The surface raster is in essence only an intermediate data format which is intended to be converted back into the vector domain by the pseuo-inverse $m^{\dagger}$.
The thought is that if we minimize the difference between $\hat{Y}$ and $Y$, we implicitly minimize the difference between $m^{\dagger}(\hat{Y})$ and $P$.
This assumes that $\mathcal{L}$ is a good loss surrogate for functional composition $d \circ m^{\dagger}$, by which we mean that $\vec{\theta}_{\mathrm{opt}}$ also is a good minimizer for:
\begin{align*}
  \sum_{(X, \mathcal{P})}
  d\left(
    m^{\dagger}(m(\mathcal{P})),~
    m^{\dagger}(\hat{f}(X; \vec{\theta}))
  \right).
\end{align*}
For a sufficiently representative raster mapping this should also imply that $\vec{\theta}_{\mathrm{opt}}$ also is a good minimizer for:
\begin{align*}
  \sum_{(X, \mathcal{P})}
  d\left(
    \mathcal{P},~
    m^{\dagger}(\hat{f}(X; \vec{\theta}))
  \right).
\end{align*}
This is the metric we really intend to minimize, but can only do so implicitly with a good surrogate loss function, and a raster format that is sufficiently representative and targetable.

\subsection{The \enquote{surface normal} raster format}

We want to construct raster arrays (tiles) which represent different types of GIS data.
Each such raster tile represents GIS data corresponding to a specific geographic area, the extent of which is specified by a bounding box $B(\vec{c}, w, h)$.
In our case we intend to construct square tiles with areas of $\SI{4096}{\meter\squared}$, i.e. having width and height $w = h = \SI{64}{\meter}$.
All of the bounding boxes have centroids $\vec{c}$ such that they are situated within the Norwegian municipality of Trondheim.
Denote the set of all these bounding boxes as $\mathcal{T}$:
%
\begin{align*}
  \mathcal{T}
  &=
  \{
    B(\vec{c}_1, \SI{64}{\meter}, \SI{64}{\meter}),
    B(\vec{c}_2, \SI{64}{\meter}, \SI{64}{\meter}),
    \dots
    B(\vec{c}_{|\mathcal{T}|}, \SI{64}{\meter}, \SI{64}{\meter})
  \}
  \\
  &=
  \{
    B_1,
    B_2,
    \dots,
    B_{|\mathcal{T}|}
  \}
\end{align*}
%
Our bounding boxes are constructed from cadastral plots in Trondheim, totaling $|\mathcal{T}| = \numtiles$.
We will restrict ourselves to constructing raster arrays of consistent resolution height $H = 256$ and resolution width $W = 256$.
This results in $256^2$ \enquote{pixelized} GIS measurements per raster tile, or equivalently, \num{16} raster pixel values per meter squared.
A specific bounding box $B(\vec{c}, \SI{64}{\meter}, \SI{64}{\meter})$ will therefore be represented by a set of $256 \times 256 \times C$ raster arrays consisting of $C$ channels.
We have already described raster arrays representing different types of GIS data, such as RGB aerial photography ($C = 3$), LiDAR elevation data ($C = 1$), or building footprint segmentation masks ($C = 1$).
The idea is now to construct an entirely new raster format which is able to represent a set of three-dimensional polygons:
\begin{equation*}
  \mathcal{P}
  =
  \{
    P_1,
    P_2,
    \dots,
    P_{|\mathcal{P}|}
  \}.
\end{equation*}
In our case $\mathcal{P}$ consists of all roof surfaces in Trondheim, totaling $|\mathcal{P}| = \numsurfaces$.
Our surface raster format, which we will refer to as the \enquote{surface normal} raster, is intended as a format which is both representative and targetable.
We start by imposing two key assumptions on the polygons contained by $\mathcal{P}$.
\begin{enumerate}
  \item\label{itm:flat} All polygons $P \in \mathcal{P}$ are \textbf{perfectly planar}.
    That is, for every polygon you can determine $\beta_0$, $\beta_x$, and $\beta_y$ such that $z = \beta_0 + \beta_x x + \beta_y y$ for \emph{all} vertices $(x, y, z)$ representing the given polygon.
  \item\label{itm:non-overlapping} All polygons $P \in \mathcal{P}$ are \textbf{mutually non-overlapping} when projected into the $xy$-plane, the $xy$-plane being the sea level.
\end{enumerate}
These assumptions are in fact \emph{not} satisfied by the Trondheim roof surface polygon dataset, but these issues are rectifiable.
Assumption \ref{itm:flat} is \textit{nearly} satisfied and can be solved by regression with negligible error.
Assumption \ref{itm:non-overlapping} can be solved with a suitable \enquote{conflict resolution} method which determines which polygon should be considered at each pixel location.
These two solutions will be described in more detail later, but for now it is easier to describe the surface normal raster with these assumptions in place.
We will introduce one final assumption:
\begin{enumerate}
  \setcounter{enumi}{2}
  \item \label{itm:simple} All polygons $P \in \mathcal{P}$ are \textbf{simple} polygons without any interior hulls, and are thus representable by a single exterior ring.
   The exterior ring is represented as an ordered sequences of $(x, y, z)$ coordinate tuples, and will therefore be denoted as,
    \begin{equation*}
      P = [(x_1, y_1, z_1), (x_2, y_2, z_2), \dots, (x_{|P|}, y_{|P|}, z_{|P|}), (x_1, y_1, z_1)],
    \end{equation*}
    where $|P|$ denotes the number of \emph{unique} vertices, and where the first and last coordinate tuples are identical in order to close the ring.
\end{enumerate}
Assumption \ref{itm:simple} is solely introduced for notational simplicity.
Keeping track of several linear rings for each polygon will substantially complicate all expressions that will follow.
There is nothing preventing the surface normal raster format for being generalized to polygons with interior hulls, and our implementation does in fact take interior hulls into account when constructing the surface normal raster.

Under the assumption of all polygons in $\mathcal{P}$ being perfectly planar we are able to decompose any polygon $P \in \mathcal{P}$ into two constituent sub-components: its \textit{two-dimensional projection} and its \textit{planar equation}.
The first sub-component is the projection of the polygon into the $xy$-plane, $\project{P}$, making the three-dimensional polygon two-dimensional instead.
This simple projection is simply performed by truncating the $z$-component of each $(x, y, z)$-vertex in the polygon:
\begin{align*}
  \project{P}
  &=
  \project{[(x_1, y_1, z_1), (x_2, y_2, z_2), \dots, (x_{|P|}, y_{|P|}, z_{|P|}), (x_1, y_1, z_1)]}
  \\
  &=
  [(x_1, y_1), (x_2, y_2), \dots, (x_{|P|}, y_{|P|}), (x_1, y_1)]).
\end{align*}
The projection mapping $\project{P}$ is also illustrated in \cref{fig:2d-polygon-projection}.
\begin{figure}
  \centering
  \includegraphics[width=0.66\linewidth]{2d-projection.tikz}
  \caption{%
    Projection of three-dimensional polygon onto $xy$-plane by $\project{\cdot}$.
  }{%
    The three-dimensional polygon is shown in \textcolor{red}{red}, while the two-dimensional projection is shown in \textcolor{blue}{blue}.
  }%
  \label{fig:2d-polygon-projection}
\end{figure}

The second sub-component is the parametric description of the plane on which \emph{all} vertices of the given polygon lie.
Denote this mapping as $\vec{\beta}(P)$ and define it according to the following relationship:
\begin{align*}
  \vec{\beta}(P)
  =
  \begin{bmatrix}
    \beta_0 \\
    \beta_x \\
    \beta_y \\
  \end{bmatrix}
  \text{ such that }
  z = \beta_0 + \beta_x x + \beta_y y \text{ for all } (x, y, z) \in P
\end{align*}
The original three-dimensional polygon can be easily reconstructed in a lossless manner from the two sub-components, as illustrated in \cref{fig:3d-polygon-reconstruction}, while still being a less redundant representation of the polygon.
%
\begin{figure}[H]
  \centering
  \includegraphics{3d-polygon-reconstruction.tikz}
  \caption{%
    The decomposition and reconstruction of a three-dimensional polygon.
  }%
  \label{fig:3d-polygon-reconstruction}
\end{figure}
\noindent
Now the idea is to create two separate rasters, one which represents $\project{P}$, and another one which represents $\vec{\beta}(P)$.
The task of rasterizing $\project{P}$ is quite simple, it is a two-dimensional polygon which can be represented as a binary mask as explained earlier.
\begin{align*}
  S_{i,j} &= \begin{cases}
    1, &\text{if there exists } P \in \mathcal{P} \text{ such that } \project{P} \text{ covers } \pixtogeo{i}{j}.\\
    0, &\text{otherwise.}
  \end{cases}
  \tag{semantic segmentation raster}
\end{align*}
Where we have defined $\pixtogeo{i}{j}$ as a function that maps array pixel coordinates $(i, j)$ to the respective coordinate in the UTM coordinate system in which the polygons $P \in \mathcal{P}$ are specified.
Given that we have a raster array of \emph{resolution} height $H$ and width $W$, which covers a geographic area represented by the bounding box $B = B(\vec{c}, w, h)$ centered at $\vec{c}$, with \emph{geographic} width $w$ and height $h$, and that $(i, j) = (0, 0)$ represents the upper left (northwestern) corner of the bounding box, then we have:
\begin{equation*}
  \pixtogeo{i}{j}
  =
  \vec{c}
  +
  \frac{1}{2} \begin{bmatrix}
    -w\\
    h
  \end{bmatrix}
  +
  \begin{bmatrix}
    \frac{w}{W} j\\
    -\frac{h}{H} i\\
  \end{bmatrix}
\end{equation*}
When it comes to the rasterization of $\planar{P} = {[\beta_0, \beta_x, \beta_y]}^T$, we start by noticing that the \textit{normal vector} of the plane can be constructed from $\planar{P}$ in the following manner:
\begin{equation*}
  \planar{P}
  =
  \begin{bmatrix}
    \beta_0\\
    \beta_x\\
    \beta_y\\
  \end{bmatrix}
  \iff
  \normalplanar{P}
  =
  \frac{1}{\sqrt{\beta_x^2 + \beta_y^2 + 1}}
  \begin{bmatrix}
    -\beta_x\\
    -\beta_y\\
    1
  \end{bmatrix}
  \defeq
  \begin{bmatrix}
    n_x\\
    n_y\\
    n_z
  \end{bmatrix}
\end{equation*}
The relationship between the equation of the plane and the associated surface normal vector is illustrated in \cref{fig:3d-normal-vector}.
\begin{figure}[H]
  \centering
  \includegraphics[width=0.55\linewidth]{3d-normal-vector.tikz}
  \caption{%
    The relationship between the equation of the plane, $\planar{P}$, and the surface normal vector, $\normalplanar{P}$.
  }{%
    The plane is defined by $z = \beta_0 + \beta_x x + \beta_y y$.
    The elements of the parameter vector $\vec{\beta}$ are shown in \textcolor{orange}{orange}, while the normal vector $\vec{n}$ is shown in \textcolor{blue}{blue}.
  }%
  \label{fig:3d-normal-vector}
\end{figure}
\noindent
The following restrictions on $\vec{n}$ hold by construction,
\begin{align*}
  ||\vec{n}||_2 = \sqrt{n_x^2 + n_y^2 + n_z^2} &\equiv 1,
  \\
  n_z &\geq 0.
\end{align*}
With other words, the normal vector has been constructed such that it is of length 1 in the Euclidean norm, and that it \emph{always} points upwards.
\noindent
Given that we can easily determine $\planar{P}$ for any polygon, we now define the following raster format:
\begin{align*}
  N_{i,j} &= \begin{cases}
    \normalplanar{P}, &\text{if $\project{P}$ covers } \pixtogeo{i}{j}.\\
    \vec{0} \defeq {[0, 0, 0]}^T, &\text{if no such } P \in \mathcal{P} \text{ exists.}
  \end{cases}
  \tag{surface normal raster}
\end{align*}
This is a raster format consisting of three raster channels as each pixel location $(i, j)$ represents a three-dimensional normal vector.
It should now become apparent why we made assumption \ref{itm:non-overlapping} earlier, that is, all polygons $P \in \mathcal{P}$ should be mutually non-overlapping after being projected by $\project{\cdot}$.
If more than one polygon covers the coordinate $\pixtogeo{i}{j}$, then the value for $N_{i,j}$ becomes ambiguous.
You may now ask why we rasterize $\mathcal{P}$ into both a surface normal raster, $N$, \emph{and} a semantic segmentation map, $S$, when $S$ can be directly inferred from N.
That is, we can formulate $S$ in form of $N$:
\begin{equation*}
  S_{i, j}
  =
  \begin{cases}
    0, &\text{if } N_{i,j} = {\left[0, 0, 0\right]}^T.  \\
    1, &\text{otherwise.}
  \end{cases}
\end{equation*}
The reason is that this is a more targetable raster decomposition.
We can now construct two relatively independent predictors, $\hat{f}_{\mathrm{seg}}(X; \vec{\theta}_{\mathrm{seg}})$ and $\hat{f}_{\mathrm{norm}}(X; \vec{\theta}_{norm})$, which when combined form a main predictor $\hat{f}$ according to:
\begin{equation*}
  \hat{f}\left(X; \vec{\theta}_{\mathrm{seg}}, \vec{\theta}_{\mathrm{norm}}\right)
  =
  \begin{cases}
    \hat{f}_{\mathrm{norm}}\left(X; \vec{\theta}_{\mathrm{norm}}\right), &\text{if } \hat{f}_{\mathrm{seg}}\left(X; \vec{\theta}_{\mathrm{seg}}\right) \geq 0.5. \\
    {\left[0, 0, 0\right]}^T, &\text{otherwise.}
  \end{cases}
\end{equation*}
Now an optimal value for $\vec{\theta}_{\mathrm{seg}}$ can be found by training $\hat{f}_{\mathrm{seg}}$ on $S$ as the ground truth, using model architectures and loss functions from the semantic segmentation literature.
A model architecture and loss function can be likewise be chosen for $\hat{f}_{\mathrm{norm}}$ \emph{entirely} independent of the segmentation problem at hand.
The normal vector model architecture can for instance enforce $||\hat{f}_{\mathrm{norm}}||_2 \equiv 1$, and the respective loss function can utilize this restriction of the model output and ground truth.
Such a \enquote{separation of concerns} has been shown to be beneficial for several model architectures.

We can now rasterize any single polygon $P \in \mathcal{P}$ into two separate raster formats, $S$ and $N$, as illustrated in \cref{fig:3d-polygon-decomposition}.
\begin{figure}
  \centering
  \includegraphics{3d-polygon-decomposition.tikz}
  \caption{The deconstruction of a three-dimensional polygon into two separate rasters formats.}
  \label{fig:3d-polygon-decomposition}
\end{figure}
When generalizing from a single three-dimensional polygon $P$ to a set of polygons $\mathcal{P}$, we can iteratively fill in the values into a single pair of raster arrays, $S$ and $N$, since the polygons are mutually non-overlapping.
In order to map from this raster domain, represented by $S$ and $N$, back to the original polygon domain, represented by $\mathcal{P}$, we propose the following pseudo-inverse mapping, $m^{\dagger}$:
\begin{framed}
  \noindent
  \textbf{Proposed $\mathbf{m^{\dagger}}$} -- Partition semantic segmentation raster $S$ such that all partitions form contiguous pixel areas which share the same value of $N$.
  Reconstruct $\beta_0$ by inference from LiDAR input data.
\end{framed}
\noindent
It should be noted that this is in fact a lossy decomposition, making the raster format not perfectly representative.
The reason for this is that the parameter $\beta_0$ has been entirely discarded when we rasterize $\planar{P}$.
The exact conditions under which the surface raster format and associated inverse mapping becomes non-representative will be discussed in detail in \cref{chap:post-processing}.
For now, suffice it to say that the following two conditions must be simultaneously satisfied.
\begin{enumerate}
  \item There exists two polygons $P_1, P_2 \in \mathcal{P}$ such that $\project{P_1}$ and $\project{P_2}$ touch borders when rasterized.
    That is, $P_1$ and $P_2$ form one single, contiguous area in the semantic segmentation map $S$.
  \item $P_1$ and $P_2$ share values for $\beta_x$ and $\beta_y$, but \emph{not} $\beta_0$.
    That is, the plane of $P_1$ and $P_2$ share the same \textit{orientation} in space, but not the same \textit{elevation}.
\end{enumerate}
In order to correct for the lossy decomposition into $S$ and $N$, we must introduce a third raster format into the mix.
Given that the pixel coordinate ${[i, j]}^T$ maps to the geographic coordinate $[x, y]^T$, i.e. $\pixtogeo{i}{j} = {[x, y]}^T$, then we define the \textit{surface elevation raster} as:
\begin{align*}
  Z_{i,j} &= \begin{cases}
    \begin{bmatrix}1, x, y\end{bmatrix} \planar{P}, &\text{if } \project{P} \text{ covers } \pixtogeo{i}{j}.\\
    -\infty, &\text{if no such } P \in \mathcal{P} \text{ exists}.
  \end{cases}
  \tag{surface elevation raster}
\end{align*}
This \textit{surface elevation} raster array enables us to define a perfectly representative raster format and associated reverse mapping.
In practice however, it will be shown that the decomposition of $\mathcal{P}$ into $S$ and $N$, discarding $Z$, is sufficient for accurate roof geometry inference.
The three surface raster formats presented so far are all illustrated in \cref{fig:interpolation-concepts}.
\begin{figure}
  \centering
  \includegraphics[width=\linewidth]{interpolation-concepts.tikz}
  \caption{Three-dimensional surface raster values}{%
    Illustration surface elevation values, $Z_{i,j}$, surface normal array values, $N_{i,j}$, and segmentation mask, $S_{i,j}$.
    Slice for $i = 1$ and $1 \leq j \leq 5$.
  }%
  \label{fig:interpolation-concepts}
\end{figure}

We will describe how to construct these three raster arrays from a practical implementation perspective.
We start by summarizing the entire implementation in conceptual terms:
\begin{leftbar}
  \noindent
  Given a set of two-dimensional bounding boxes $\mathcal{T}$ and a set of three-dimensional polygons $\mathcal{P}$:
  \begin{itemize}[nosep,leftmargin=*]
    \item Construct R-tree index for polygon collection $\mathcal{P}$.
    \item Calculate and memoize $\planar{P}$ for all $P \in \mathcal{P}$.
    \item For each bounding box $B \in \mathcal{T}$\ldots
    \begin{itemize}[nosep,leftmargin=0.5cm]
      \item Determine the subset of polygons $\mathcal{P}_B \subset \mathcal{P}$ which is at least partially covered by the bounding box $B$.
        The aforementioned R-tree index is used in order to substantially speed up this spatial query.
      \item Map all coordinates of all $P \in \mathcal{P}_B$ to the pixel coordinate system $\left[0, 255\right] \times \left[0, 255\right]$.
      \item For each pixel coordinate $(i, j)$\ldots
      \begin{itemize}[nosep,leftmargin=0.5cm]
        \item Determine subset of polygons $\mathcal{P}_{\mathrm{cover}} \subset \mathcal{P}_B$ which covers the area represented by the pixel $(i, j)$.
          If $|\mathcal{P}_{\mathrm{cover}}| = 0$, set $Z_{i,j} = -\infty$, $N_{i, j} = {[0, 0, 0]}^T$, and $S_{i, j} = 0$, and continue onto next iteration of loop.
        \item Fetch pre-calculated values for $\beta(P)$ for all $P \in \mathcal{P}_{\mathrm{cover}}$.
        \item Given that $\pixtogeo{i}{j} = [x,~y]^T$, calculate surface elevation $z_P = \beta_0 + \beta_x x + \beta_y y$ for all $P \in \mathcal{P}_{\mathrm{cover}}$.
          Select polygon $P_m$ with the greatest corresponding elevation value $z_P$.
        \item Set $Z_{i,j} = z_{P_m}$, $N_{i, j} = \normalplanar{P_m}$, and $S_{i, j} = 1$.
      \end{itemize}
    \end{itemize}
  \end{itemize}
\end{leftbar}
\noindent
All of these steps will now be explained in more detail.

\subsubsection{R-Tree index}

We will construct a computational procedure which is able to rasterize a given polygon $P \in \mathcal{P}$ relative to a given bounding box $B$.
Denote this procedure as $\texttt{Rasterize}(P,~B)$, and say that this procedure returns three rasters: a surface elevation raster $Z_p$, a surface normal raster $N_p$, and a two-dimensional segmentation mask $S_p$.
The idea is to iterate over the polygons in $\mathcal{P}$, rasterize them individually with \texttt{Rasterize()}, and fill these arrays into mutable arrays $Z$, $N$, and $S$ with the procedure \texttt{FillValues()}, defined as follows:
%
\begin{pseudofunc}{FillValues}{$\texttt{from}=[Z_p, N_p, S_p],~\texttt{into}=[Z, N, S]$}
  \item for $i \in \{0, 1, \ldots, 255\}$:
  \begin{pseudoloop}
    \item for $j \in \{0, 1, \ldots, 255\}$:
    \begin{pseudoloop}
      \item If $(S_p)_{i,j} = 0$: continue onto next iteration of inner loop
      \item Insert $S_{i,j} \leftarrow 1$, $N_{i,j} \leftarrow (N_p)_{i,j}$, and $Z_{i,j} \leftarrow (Z_p)_{i,j}$
    \end{pseudoloop}
  \end{pseudoloop}
\end{pseudofunc}
%
Now, the naive implementation of the surface rasterization preprocessing procedure can be written as follows,
%
\begin{pseudofunc}{PreprocessRaster}{$\mathcal{P},~\mathcal{T}$}
  \item for $B$ in $\mathcal{T}$:
  \begin{pseudoloop}
    \item $Z, N, S \leftarrow \texttt{ConstructPlaceHolderArrays()}$
    \item for $P$ in $\mathcal{P}$:
    \begin{pseudoloop}
      \item $Z_p, N_p, S_p \leftarrow \texttt{Rasterize}(P,~B)$
      \item $\texttt{FillValues(from}=[Z_p, N_p, S_p],~\texttt{into}=[Z, N, S])$
    \end{pseudoloop}
    \item $\texttt{SaveToHashLookup(data}=[Z, N, S],~\texttt{hash}=B)$.
  \end{pseudoloop}
\end{pseudofunc}
%
\texttt{SaveToHashLookup()} is a procedure which persists the arrays $Z$, $N$, and $S$, to disk for later retrieval.
Assume \texttt{Rasterize} to run in constant computational time, that is $\texttt{Rasterize()} = \bigo{1}$ (which is close to the truth, at least sufficiently so for this analysis).
This results in the time complexity of \texttt{PreprocessRaster()} being $\bigo{|\mathcal{P}| |\mathcal{T}|}$.
In our case, using bounding boxes and roof surface polygons from Trondheim, we have $|\mathcal{P} = \numsurfaces$ and $|\mathcal{T}| = \numtiles$.
We can therefore safely conclude that $|\mathcal{P}||\mathcal{T}| = \numcombinations \gg 1$ is infeasible for a procedure of time complexity $\bigo{|\mathcal{P}| |\mathcal{T}|}$.
A better approach is to reduce the number of iterations performed by the inner loop over $P \in \mathcal{P}$, utilizing the fact that only polygons $P \in \mathcal{P}$ for which $\project{P}$ intersects with $B$ will ever contribute with values to the raster arrays.
We define such a filter procedure \texttt{IntersectingFilter()}:
%
\begin{pseudofunc}{IntersectFilter}{$\mathcal{P},~B$}
  \item return $\left\{P \mid P \in \mathcal{P} \text{ and } \project{P} \text{ intersects with } B \right\}$
\end{pseudofunc}
%
Using \texttt{IntersectingFilter()} we can improve upon \texttt{PreprocessRaster()} as follows:
%
\begin{pseudofunc}{PreprocessRaster}{$\mathcal{P},~\mathcal{T}$}
  \item for $B$ in $\mathcal{T}$:
  \begin{pseudoloop}
    \item $Z, N, S \leftarrow \texttt{ConstructPlaceHolderArrays()}$
    \item $\mathcal{P}_B \leftarrow \texttt{IntersectingFilter}(\mathcal{P},~B)$
    \item for $P$ in $\mathcal{P}_B$:
    \begin{pseudoloop}
      \item $Z_p, N_p, S_p \leftarrow \texttt{Rasterize}(P,~B)$
      \item $\texttt{FillValues(from}=[Z_p, N_p, S_p],~\texttt{into}=[Z, N, S])$
    \end{pseudoloop}
    \item $\texttt{SaveToHashLookup(data}=[Z, N, S],~\texttt{hash}=B)$.
  \end{pseudoloop}
\end{pseudofunc}
%
If we denote the \textit{average} size of the set $\mathcal{P}_B$ as $\overline{|\mathcal{P}_B|}$, that is,
%
\begin{equation*}
  \overline{|\mathcal{P}_B|}
  =
  \frac{1}{|\mathcal{T}|}
  \sum_{B \in \mathcal{T}}
  \big|
    \left\{P \mid P \in \mathcal{P} \text{ and } \project{P} \text{ intersects with } B \right\}
  \big|,
\end{equation*}
%
then the time complexity of this new implementation of \texttt{PreprocessRaster()} becomes $\bigo{|\mathcal{T}|\overline{|\mathcal{P}_B|}}$.
In fact, for our dataset $\overline{|\mathcal{P}_B|} \approx \num{28.4}$ and $|\mathcal{T}|\overline{|\mathcal{P}_B|} = \num{1822376}$, the full distribution being plotted in \cref{fig:tile-surface-distribution}.
\begin{figure}
  \includegraphics{tile-surface-distribution}
  \caption{%
    Distribution of number of surface polygons contained by each raster tile.
  }%
  \label{fig:tile-surface-distribution}
\end{figure}
This is a great improvement, but we have neglected the time complexity of $\texttt{IntersectingFilter}(\mathcal{P}, B)$, which if linear or super-linear in $|\mathcal{P}|$ brings us back to a total time complexity of $\bigo{|\mathcal{T}||\mathcal{P}|}$.
In practice, however, the constant cost term is much smaller, and the implementation becomes much faster as a result.
However, if $\texttt{IntersectingFilter}()$ could be made close to a constant time operation, $\bigo{1}$, then we would truly be able to construct a close to $\bigo{|\mathcal{T}|\overline{|\mathcal{P}_B|}}$ procedure.
The solution is to pre-compute a so-called R-tree spatial index, which when first computed, allows for almost instantaneous intersection filtering\cite{rtree}.
\begin{pseudofunc}{PreprocessRaster}{$\mathcal{P},~\mathcal{T}$}
  \item $\texttt{RTreeIndex} \leftarrow \texttt{GenerateRTreeIndex}(\mathcal{P})$
  \item for $B$ in $\mathcal{T}$:
  \begin{pseudoloop}
    \item $Z, N, S \leftarrow \texttt{ConstructPlaceHolderArrays()}$
    \item $\hat{\mathcal{P}}_B \leftarrow \texttt{RTreeIndex}(B)$
    \item for $P$ in $\hat{\mathcal{P}}_B$:
    \begin{pseudoloop}
      \item $Z_p, N_p, S_p \leftarrow \texttt{Rasterize}(P,~B)$
      \item $\texttt{FillValues(from}=[Z_p, N_p, S_p],~\texttt{into}=[Z, N, S])$
    \end{pseudoloop}
    \item $\texttt{SaveToHashLookup(data}=[Z, N, S],~\texttt{hash}=B)$.
  \end{pseudoloop}
\end{pseudofunc}

\begin{equation*}
  \texttt{IntersectingFilter}(\mathcal{P},~B)
  \subseteq
  \texttt{GenerateRTreeIndex}(\mathcal{P})(B)
\end{equation*}

\subsubsection{Linear regression}

Now assume that the polygon collection at hand contains three-dimensional polygons which are all approximately planar.
Any $(x, y, z)$ vertex must therefore satisfy the following relationship:
%
\begin{equation*}
  z = \beta_{p,0} + \beta_{p,x} x + \beta_{p,y} y + \varepsilon
\end{equation*}
%
Where $\varepsilon$ is error term due to measurement errors or other type of random data errors.
The distribution of $\varepsilon$ must be investigated further for our dataset, but for the moment assume the error to be normally distributed with zero mean and some unknown variance $\sigma^2$, i.e. $\varepsilon \sim \mathcal{N}(0, \sigma^2)$.
%
The task is now to determine the coefficient vector $\vec{\beta}_p = {[\beta_{p,0}, \beta_{p,x}, \beta_{p,y}]}^T$ which describes the planar polygon surface.
We construct a design matrix $X_p$ consisting of all $(x, y)$ vertex coordinate tuples of the given polygon $P_p$:
%
\begin{equation*}
  X_p
  =
  \begin{bmatrix}
    1 & x_{p,1,1} & y_{p,1,1} \\
    1 & x_{p,1,2} & y_{p,1,2} \\
    \vdots & \vdots & \vdots \\
    1 & x_{p,1,|r_{p,1}|} & y_{p,1,|r_{p,1}|} \\
    1 & x_{p,2,1} & y_{p,2,1} \\
    1 & x_{p,2,2} & y_{p,2,2} \\
    \vdots & \vdots & \vdots \\
    1 & x_{p,|P_p|,|r_{|P_p|}|} & y_{p,|P_p|,|r_{|P_p|}|} \\
  \end{bmatrix}.
\end{equation*}
%
Likewise, a response vector $\vec{z}_p$ is constructed consisting of the respective elevation values associated with the $(x, y)$ tuples:
%
\begin{equation*}
  \vec{z}_p
  =
  \begin{bmatrix}
     z_{p,1,1} \\
     z_{p,1,2} \\
     \vdots \\
     z_{p,1,|r_{p,1}|} \\
     z_{p,2,1} \\
     z_{p,2,2} \\
     \vdots \\
     z_{p,|P_p|,|r_{|P_p|}|} \\
  \end{bmatrix}
\end{equation*}
%
Again, by assuming $\varepsilon \sim \mathcal{N}(0, \sigma^2)$ we construct an ordinary least squares estimator $\widehat{\vec{\beta}}_p $ for $\vec{\beta}_p$:
%
\begin{equation*}
  \widehat{\vec{\beta}}_p
  =
  {\left[
    \widehat{\beta}_{p,0},
    \widehat{\beta}_{p,x},
    \widehat{\beta}_{p,y}
  \right]}^T
  =
  \left( X_p^T X_p \right)^{-1} X_p^T \vec{z}_p
\end{equation*}
%
Interpolated elevation values for arbitrary $(x, y)$ coordinate tuples can now be constructed:
%
\begin{equation*}
  \widehat{z} = \widehat{\beta}_{p,0} + \widehat{\beta}_{p,x} x + \widehat{\beta}_{p,y} y
\end{equation*}
%
Or as an alternative formulation, a linear predictor $\widehat{f}_z$ parametrized according to $\widehat{\vec{\beta}}_p$ can be constructed:
%
\begin{equation*}
  \widehat{f}_z\left(x, y; \widehat{\vec{\beta}}_p\right)
  =
    \widehat{\beta}_{p,0}
    + \widehat{\beta}_{p,x} x
    + \widehat{\beta}_{p,y} y
  =
  \widehat{z}
\end{equation*}
%
Now define the coefficient vector set $\mathcal{B}(x, y)$ which contains all coefficient vectors to polygons that contain the coordinate $(x, y)$ when projected into the $xy$-plane.
More formally,
%
\begin{equation*}
  \mathcal{B}(x, y) = \left\{
    \vec{\beta}_p
    \mid
    (x, y) \in \project{P_p}
  \right\},
\end{equation*}
%
where $\project{P_p}$ is the projection of all polygon vertex coordinates into the $xy$-plane as illustrated in \cref{fig:2d-polygon-projection}.
%
\begin{equation*}
  \vec{\beta}_m(x, y)
  =
  \argmax_{\vec{\beta} \in \mathcal{B}(x, y)}
    \widehat{f}_z(x, y; \vec{\beta})
\end{equation*}
%
\begin{equation*}
  \vec{n}\left(\vec{\beta}\right)
  =
  \frac{%
    1
  }{%
    \sqrt{\beta_x^2 + \beta_y^2 + 1}
  }
  \cdot
  {\left[
    -\beta_{x}, -\beta_{y}, 1
  \right]}^T
\end{equation*}



\begin{algorithm}{Surface interpolation}{alg:surface-interpolation}{Tile bounding box $B(\vec{c}, w, h)$,\\Raster dimensions $H \times W$,\\R-Tree indexed polygon collection.}
\item Construct arrays with temporary placeholder values:
  \begin{itemize}[label=--,leftmargin=0cm]
    \item Normal vector array $N$ of size $H \times W \times 3$ filled with $0$.
    \item Interpolated elevation array $Z$ of size $H \times W \times 1$ filled with $-\infty$.
  \end{itemize}
\item Construct polygon collection $\mathcal{P}$ for given tile extents using R-Tree index.
  Convert polygons to pixel coordinate system.
\item For polygon $P_p = [r_{p0}, r_{p1}, \dots, r_{pn_p}]$ with a single exterior ring, $r_{p0}$, and $n_p$ interior rings, $[r_{p1}, \ldots, r_{pn_p}]$, in polygon collection $\mathcal{P}$.
  \begin{enumerate}[leftmargin=0.5em,label=\textbf{\alph*}]
      % \item The linear ring $r_{pj} = [(x_{pj0}, y_{pj0}, z_{ij0}), \dots, (x_{pjn_{ij}}, y_{ijn_{pj}}, z_{ijn_{pj}})]$.
    \item Construct $M \times 3$ design matrix $X$ populated with pixel coordinates of all \textit{unique} xy-vertices $(1, x_{pij}, y_{pij})$.
        Secondly, construct $M \times 1$ response vector $\vec{y}$ populated with the respective $z_{pij}$ coordinates.
    \item Solve linear regression problem $\vec{\beta}_p = {\left(\beta_0, \beta_x, \beta_y\right)}^T = {\left(X^T X\right)}^{-1} X^T \vec{y}$
    \item Construct polygon surface normal vector $\vec{n}_p$,
      \begin{equation*}
        \vec{n}_p \assign {\left(\beta_x^2 + \beta_y^2 + 1\right)}^{-1/2} \cdot {\left(-\beta_x, -\beta_y, 1\right)}^T,
      \end{equation*}
      where $||\vec{n}_p||_2 = 1$ by construction.
    \item For each pixel coordinate $(i, j)$ contained by the polygon $P_p$ projected onto the 2-dimensional pixel index plane:
      \begin{itemize}[leftmargin=0.5em]
        \item $h \assign \beta_0 + \beta_x j + \beta_y i$. If $h < Z_ij$, continue loop, else\dots
        \item $Z_{ij} \assign h$ and $\left(N_{ijx}, N_{ijy}, N_{ijz}\right) \assign \vec{n}_p$.
      \end{itemize}
    \end{enumerate}
\end{algorithm}
