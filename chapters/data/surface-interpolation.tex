We want to construct rasterized data tiles which geographic extent are specified by bounding boxes $B(\vec{c}, w, h)$.
Specifically we construct square tiles of area $\SI{4096}{\meter\squared}$, i.e. $w = h = \SI{64}{\meter}$, which will contain \num{4} raster values per meter, or equivalently, \num{16} raster pixels values per meter squared.
A specific bounding box $B(\vec{c}, \SI{64}{\meter}, \SI{64}{\meter})$ will therefore be represented by a set of $256 \times 256 \times C$ raster arrays consisting of $C$ channels.
These raster arrays will represent different types of GIS data such as RGB aerial photography ($C = 3$), LiDAR elevation data ($C = 1$), or building footprint segmentation masks ($C = 1$).
The idea is now to introduce two new raster array types constructed from three-dimensional surface polygons; a $256 \times 256 \times 3$ \textit{surface normal array} $N$ and a $256 \times 256 \times 1$ \textit{surface elevation array} $H$.
The interpretation and construction of these arrays will be explained in this section.
The dataset used in this thesis contains all roof surfaces in the Norwegian municipality of Trondheim.
For now assume that this collection of polygons represents perfectly flat surfaces in three dimensions, all of which are mutually non-overlapping even when projected onto the $xy$-plane.
%
\begin{align*}
  N_{i,j} &= \begin{cases}
    \text{surface normal vector,} &\text{if surface exists at } (i, j).\\
    \vec{0}, &\text{otherwise.}
  \end{cases}
  \\
  Z_{i,j} &= \begin{cases}
    \text{surface elevation at } $(i, j)$ &\text{if surface exists at } (i, j).\\
    -\infty, &\text{otherwise.}
  \end{cases}
\end{align*}
%
In the case of overlapping polygons the choice of polygon to use in order to fill surface normal and elevation values at overlapped coordinates becomes ambiguous.
In the case of polygons which are not perfectly flat, meaning that all vertices are \emph{not} situated perfectly on the same geometric plane, the interpolation of elevation values to the interior of the polygon becomes ambiguous as well.

Our data set consists of a set of flat, planar, three-dimensional polygons.

\begin{figure}
  \centering
  \includegraphics[width=\linewidth]{interpolation-concepts}
  \caption{Three-dimensional surface raster values}{%
    Illustration of interpolated altitude values, $Z_{ij}$, and respective surface normal vectors, $N_{ij}$.
    Slice for $i = 1$ and $1 \leq j \leq 4$.
  }
\end{figure}

\begin{align*}
  N: i \times j \rightarrow \vec{n},
  \hspace{2em} & \text{ where } \vec{n} \in \mathbb{R}^3, ||\vec{n}||_2 \in \{0, 1\},
  \text{ and } i, j \in [0, 255].
\end{align*}


\begin{leftbar}
  \begin{itemize}[nosep,leftmargin=*]
    \item Determine geometric plane for all polygons in polygon collection.
    \item For each tile bounding box\ldots
    \begin{itemize}[nosep,leftmargin=0.5cm]
      \item Determine the subset of polygons which intersects the tile bounding box.
      \item Map polygon subset from geographic coordinate system to pixel coordinate system.
      \item For each pixel in rasterized tile\ldots
      \begin{itemize}[nosep,leftmargin=0.5cm]
        \item Determine subset of polygons which intersects the pixel area.
        \item If more than one polygon covers the given pixel coordinate, choose the polygon with the greatest elevation value at that given point.
      \end{itemize}
    \end{itemize}
  \end{itemize}
\end{leftbar}

\subsubsection{Notation}

Let $r_{p,i}$ denote a linear ring belonging to a polygon denoted as $P_p$.
The symbol $p$ denotes the \textit{polygon index}, while $i$ denotes the \textit{ring index}.
A ring index of 1 indicates an \textit{exterior ring}, while a ring index of greater than one represents an interior ring.
The polygon $P_p$ can therefore be represented as a sequence of $|P_p| \geq 1$ linear rings,
%
\begin{equation*}
P_p = \{ r_{p,1}, \dots, r_{p, |P_p|}\}
\end{equation*}
%
Linear rings are represented as ordered sequences of $(x, y, z)$ coordinate tuples, assuming that the associated polygon is three-dimensional.
The linear ring can therefore be denoted as
%
\begin{equation*}
  r_{p,i}
  =
  \{
    (x_{p,i,1}, y_{p,i,1}, z_{p,i,1}),
    \dots,
    (x_{p,i,|r_{p, i}|}, y_{p,i,|r_{p, i}|}, z_{p,i,|r_{p, i}|}),
    (x_{p,i,1}, y_{p,i,1}, z_{p,i,1})
  \},
\end{equation*}
%
where the first and last coordinate tuple are identical in order to close the ring.

\subsubsection{R-Tree index}

Denote the indexed collection of three-dimensional polygons $\mathcal{P}$, and assume that the size of the collection is of size $|\mathcal{P}| \gg 1$, i.e. $\mathcal{P} = \{P_1, P_2, \dots, P_{|\mathcal{P}|} \}$.
In our case the collection $\mathcal{P}$ consists of all roof surfaces in the municipality of Trondheim, Norway with $|\mathcal{P}| = \num{293336}$.
We want construct a set of rasterized tiles defined over a respective set of bounding boxes $\mathcal{T} = \{B(\vec{c}_1, \SI{64}{\meter}, \SI{64}{\meter}), B(\vec{c}_2, \SI{64}{\meter}, \SI{64}{\meter}),\allowbreak \dots,\allowbreak B(\vec{c}_{|\mathcal{T}|}, \SI{64}{\meter}, \SI{64}{\meter}) \}$ based on the polygon collection $\mathcal{P}$.
Denote the algorithmic procedure which construct rasterized tiles \texttt{ConstructTile}.
Assume that we \texttt{ConstructTile} and that this procedure accepts a bounding box $B(\vec{c}, w, h)$ and a polygon collection $\mathcal{P}$ as inputs.
The average time complexity of \texttt{ConstructTile} will most likely be at least linearly dependent of the polygon collection size $|\mathcal{P}|$.
%
\begin{equation*}
  \mathcal{P} \cap B(\vec{c}, h, w)
  =
  \left\{
    P \in \mathcal{P} \mid P \cap B(\vec{c}, h, w) \neq \varnothing
  \right\}
\end{equation*}
%
\begin{equation*}
  \texttt{ConstructTile}(B(\vec{c}, h, w),~\mathcal{P})
  \equiv
  \texttt{ConstructTile}(B(\vec{c}, h, w),~\mathcal{P} \cap B(\vec{c}, h, w))
\end{equation*}

\subsubsection{Linear regression}

Now assume that the polygon collection at hand contains three-dimensional polygons which are all approximately planar.
Any $(x, y, z)$ vertex must therefore satisfy the following relationship:
%
\begin{equation*}
  z = \beta_{p,0} + \beta_{p,x} x + \beta_{p,y} y + \varepsilon
\end{equation*}
%
Where $\varepsilon$ is error term due to measurement errors or other type of random data errors.
The distribution of $\varepsilon$ must be investigated further for our dataset, but for the moment assume the error to be normally distributed with zero mean and some unknown variance $\sigma^2$, i.e. $\varepsilon \sim \mathcal{N}(0, \sigma^2)$.
%
The task is now to determine the coefficient vector $\vec{\beta}_p = {[\beta_{p,0}, \beta_{p,x}, \beta_{p,y}]}^T$ which describes the planar polygon surface.
We construct a design matrix $X_p$ consisting of all $(x, y)$ vertex coordinate tuples of the given polygon $P_p$:
%
\begin{equation*}
  X_p
  =
  \begin{bmatrix}
    1 & x_{p,1,1} & y_{p,1,1} \\
    1 & x_{p,1,2} & y_{p,1,2} \\
    \vdots & \vdots & \vdots \\
    1 & x_{p,1,|r_{p,1}|} & y_{p,1,|r_{p,1}|} \\
    1 & x_{p,2,1} & y_{p,2,1} \\
    1 & x_{p,2,2} & y_{p,2,2} \\
    \vdots & \vdots & \vdots \\
    1 & x_{p,|P_p|,|r_{|P_p|}|} & y_{p,|P_p|,|r_{|P_p|}|} \\
  \end{bmatrix}.
\end{equation*}
%
Likewise, a response vector $\vec{z}_p$ is constructed consisting of the respective elevation values associated with the $(x, y)$ tuples:
%
\begin{equation*}
  \vec{z}_p
  =
  \begin{bmatrix}
     z_{p,1,1} \\
     z_{p,1,2} \\
     \vdots \\
     z_{p,1,|r_{p,1}|} \\
     z_{p,2,1} \\
     z_{p,2,2} \\
     \vdots \\
     z_{p,|P_p|,|r_{|P_p|}|} \\
  \end{bmatrix}
\end{equation*}
%
Again, by assuming $\varepsilon \sim \mathcal{N}(0, \sigma^2)$ we construct an ordinary least squares estimator $\widehat{\vec{\beta}}_p $ for $\vec{\beta}_p$:
%
\begin{equation*}
  \widehat{\vec{\beta}}_p
  =
  {\left[
    \widehat{\beta}_{p,0},
    \widehat{\beta}_{p,x},
    \widehat{\beta}_{p,y}
  \right]}^T
  =
  \left( X_p^T X_p \right)^{-1} X_p^T \vec{z}_p
\end{equation*}
%
Interpolated elevation values for arbitrary $(x, y)$ coordinate tuples can now be constructed:
%
\begin{equation*}
  \widehat{z} = \widehat{\beta}_{p,0} + \widehat{\beta}_{p,x} x + \widehat{\beta}_{p,y} y
\end{equation*}
%
Or as an alternative formulation, a linear predictor $\widehat{f}_z$ parametrized according to $\widehat{\vec{\beta}}_p$ can be constructed:
%
\begin{equation*}
  \widehat{f}_z\left(x, y; \widehat{\vec{\beta}}_p\right)
  =
    \widehat{\beta}_{p,0}
    + \widehat{\beta}_{p,x} x
    + \widehat{\beta}_{p,y} y
  =
  \widehat{z}
\end{equation*}
%
Now define the coefficient vector set $\mathcal{B}(x, y)$ which contains all coefficient vectors to polygons that contain the coordinate $(x, y)$ when projected into the $xy$-plane.
More formally,
%
\begin{equation*}
  \mathcal{B}(x, y) = \left\{
    \vec{\beta}_p
    \mid
    (x, y) \in \pi_{\mathrm{2D}}(P_p)
  \right\},
\end{equation*}
%
where $\pi_{\mathrm{2D}}$ is the projection of all polygon vertex coordinates into the $xy$-plane as illustrated in \cref{fig:2d-polygon-projection}.
%
\begin{figure}
  \centering
  \includegraphics[width=\linewidth]{2d-projection}
  \caption{%
    Projection of three-dimensional polygon onto $xy$-plane by $\pi_{\mathrm{2D}}$.
  }{%
    The three-dimensional polygon is shown in \textcolor{red}{red}, while the two-dimensional projection is shown in \textcolor{blue}{blue}.
  }%
  \label{fig:2d-polygon-projection}
\end{figure}
%
\begin{equation*}
  \vec{\beta}_m(x, y)
  =
  \argmax_{\vec{\beta} \in \mathcal{B}(x, y)}
    \widehat{f}_z(x, y; \vec{\beta})
\end{equation*}
%
\begin{equation*}
  \vec{n}\left(\vec{\beta}\right)
  =
  \frac{%
    1
  }{%
    \sqrt{\beta_x^2 + \beta_y^2 + 1}
  }
  \cdot
  {\left[
    -\beta_{x}, -\beta_{y}, 1
  \right]}^T
\end{equation*}



\begin{algorithm}{Surface interpolation}{alg:surface-interpolation}{Tile bounding box $B(\vec{c}, w, h)$,\\Raster dimensions $H \times W$,\\R-Tree indexed polygon collection.}
\item Construct arrays with temporary placeholder values:
  \begin{itemize}[label=--,leftmargin=0cm]
    \item Normal vector array $N$ of size $H \times W \times 3$ filled with $0$.
    \item Interpolated elevation array $Z$ of size $H \times W \times 1$ filled with $-\infty$.
  \end{itemize}
\item Construct polygon collection $\mathcal{P}$ for given tile extents using R-Tree index.
  Convert polygons to pixel coordinate system.
\item For polygon $P_p = [r_{p0}, r_{p1}, \dots, r_{pn_p}]$ with a single exterior ring, $r_{p0}$, and $n_p$ interior rings, $[r_{p1}, \ldots, r_{pn_p}]$, in polygon collection $\mathcal{P}$.
  \begin{enumerate}[leftmargin=0.5em,label=\textbf{\alph*}]
      % \item The linear ring $r_{pj} = [(x_{pj0}, y_{pj0}, z_{ij0}), \dots, (x_{pjn_{ij}}, y_{ijn_{pj}}, z_{ijn_{pj}})]$.
    \item Construct $M \times 3$ design matrix $X$ populated with pixel coordinates of all \textit{unique} xy-vertices $(1, x_{pij}, y_{pij})$.
        Secondly, construct $M \times 1$ response vector $\vec{y}$ populated with the respective $z_{pij}$ coordinates.
    \item Solve linear regression problem $\vec{\beta}_p = {\left(\beta_0, \beta_x, \beta_y\right)}^T = {\left(X^T X\right)}^{-1} X^T \vec{y}$
    \item Construct polygon surface normal vector $\vec{n}_p$,
      \begin{equation*}
        \vec{n}_p \assign {\left(\beta_x^2 + \beta_y^2 + 1\right)}^{-1/2} \cdot {\left(-\beta_x, -\beta_y, 1\right)}^T,
      \end{equation*}
      where $||\vec{n}_p||_2 = 1$ by construction.
    \item For each pixel coordinate $(i, j)$ contained by the polygon $P_p$ projected onto the 2-dimensional pixel index plane:
      \begin{itemize}[leftmargin=0.5em]
        \item $h \assign \beta_0 + \beta_x j + \beta_y i$. If $h < Z_ij$, continue loop, else\dots
        \item $Z_{ij} \assign h$ and $\left(N_{ijx}, N_{ijy}, N_{ijz}\right) \assign \vec{n}_p$.
      \end{itemize}
    \end{enumerate}
\end{algorithm}
