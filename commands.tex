\newcommand{\NTNU}{{N}orwegian {U}niversity of {S}cience and {T}echnology}
\newcommand{\NTNUTrondheim}{{NTNU} in {T}rondheim}

% Vector notation as bold italic symbols
\renewcommand{\vec}[1]{\boldsymbol{#1}}

% Assign value to variable, mostly used in algorithm environment
\newcommand{\assign}[0]{\leftarrow}

% Requires amsmath package
\DeclareMathOperator*{\argmin}{\arg\!\min}
\DeclareMathOperator*{\argmax}{\arg\!\max}

% Todo comment
\newcommand{\todo}[1]{\textcolor{red}{\textbf{TODO:} #1}}

% Defined as equal to, ":="
\newcommand{\defeq}{\vcentcolon=}

% The decomposition of a three-dimensional polygon to two dimensions
\newcommand{\project}[1]{\pi_{\mathrm{2D}}\left(#1\right)}
% The decomposition of a three-dimensional polygon to its planar equation
\newcommand{\planar}[1]{\vec{\beta}\left(#1\right)}
% The (lossy) decomposition of a three-dimensional polygon to its normal vector
\newcommand{\normalplanar}[1]{\vec{n}\left(\planar{#1}\right)}
% Ordinary least squares predictior for the planar equation
\newcommand{\planarestimator}[1]{\widehat{\vec{\beta}}\left(#1\right)}

% Mapping from pixel coordinate to real geographic coordinate
\newcommand{\pixtogeo}[2]{\pi_{B}\left(#1,~#2\right)}

% Big-O notation
\newcommand{\bigo}[1]{\mathcal{O}\left(#1\right)}

% Double bar norm
\newcommand{\norm}[1]{\left\lVert#1\right\rVert}

% Provide relative coordinate in tikz node
\newcommand{\relcoord}[3]{($(#1.south west)!#3!(#1.north west) + (#1.south west)!#2!(#1.south east) - (#1.south west)$)}

% Different types of models, rasters, and so on
\newcommand{\segmodel}[0]{\widehat{f}_{\mathrm{seg}}}
\newcommand{\normmodel}[0]{\widehat{f}_{\mathrm{norm}}}
\newcommand{\segraster}[0]{S}
\newcommand{\predsegraster}[0]{\widehat{S}}
\newcommand{\predbinarysegraster}[0]{\widetilde{S}}
\newcommand{\normraster}[0]{N}
\newcommand{\predsegnormraster}[0]{\widetilde{N}}
\newcommand{\prednormraster}[0]{\widehat{N}}
\newcommand{\rgbraster}[0]{I}
\newcommand{\lidarraster}[0]{H}
\newcommand{\polygons}[0]{\mathcal{P}}
\newcommand{\predpolygons}[0]{\widehat{\mathcal{P}}}
\newcommand{\pseudoinverse}[0]{m^{\dagger}}
\newcommand{\rasterdomain}[0]{R}
\newcommand{\vectordomain}[0]{V}

% The first argument becomes the short caption, while the second argument appended to the first becomes the long one
\newcommand{\appcaption}[2]{\caption[#1]{#1 #2}}
